\documentclass[12pt,letterpaper,reqno]{article}

% \usepackage{mathtools}
\usepackage{epsfig}
\usepackage{amsmath}
\usepackage{amssymb}
\usepackage{amsthm}
\usepackage{indentfirst}
\usepackage{xspace}
\usepackage{multirow}
\usepackage{hyperref}
\usepackage{xcolor}
\usepackage{verbatim}
\usepackage[letterpaper,margin=1in,headheight=15pt]{geometry}
\usepackage{mathpazo}
\usepackage{tikz-cd}
\usepackage{booktabs}
\usepackage{framed}
\usepackage{float}
\usepackage{thmtools}
\usepackage{dashrule}
\usepackage[missing=]{gitinfo2}
\usepackage{fancyhdr}
\usepackage{enumerate}
\usepackage{graphicx}
\usepackage{mathrsfs}
\usepackage{calligra}
\usepackage[titletoc,title]{appendix}

\definecolor{darkblue}{rgb}{0.1,0.1,0.7}
\definecolor{darkred}{rgb}{0.5,0.1,0.1}
\definecolor{darkgreen}{rgb}{0.0,0.42,0.06}
\hypersetup{colorlinks=true,urlcolor=darkred,linkcolor=darkblue,citecolor=darkred}
\definecolor{shadecolor}{rgb}{0.85,0.85,0.85}

% Bibliography formatting
\usepackage[bibstyle=authoryear-comp,labeldate=false,defernumbers=true,maxnames=20,uniquename=init,dashed=false,backend=biber,sorting=none]{biblatex}

\DeclareNameAlias{sortname}{first-last}

\DeclareFieldFormat{url}{\url{#1}}
\DeclareFieldFormat[article]{pages}{#1}
\DeclareFieldFormat[inproceedings]{pages}{\lowercase{pp.}#1}
\DeclareFieldFormat[incollection]{pages}{\lowercase{pp.}#1}
\DeclareFieldFormat[article]{volume}{\textbf{#1}}
\DeclareFieldFormat[article]{number}{(#1)}
\DeclareFieldFormat[article]{title}{\MakeCapital{#1}}
\DeclareFieldFormat[inproceedings]{title}{#1}
\DeclareFieldFormat{shorthandwidth}{#1}

% Don't use "In:" in bibliography. Omit urls from journal articles.
\DeclareBibliographyDriver{article}{%
  \usebibmacro{bibindex}%
  \usebibmacro{begentry}%
  \usebibmacro{author/editor}%
  \setunit{\labelnamepunct}\newblock
  \MakeSentenceCase{\usebibmacro{title}}%
  \newunit
  \printlist{language}%
  \newunit\newblock
  \usebibmacro{byauthor}%
  \newunit\newblock
  \usebibmacro{byeditor+others}%
  \newunit\newblock
  \printfield{version}%
  \newunit\newblock
%  \usebibmacro{in:}%
  \usebibmacro{journal+issuetitle}%
  \newunit\newblock
  \printfield{note}%
  \setunit{\bibpagespunct}%
  \printfield{pages}
  \newunit\newblock
  \usebibmacro{eprint}
  \newunit\newblock
  \printfield{addendum}%
  \newunit\newblock
  \usebibmacro{pageref}%
  \usebibmacro{finentry}}

% Remove dot between volume and number in journal articles.
\renewbibmacro*{journal+issuetitle}{%
  \usebibmacro{journal}%
  \setunit*{\addspace}%
  \iffieldundef{series}
    {}
    {\newunit
     \printfield{series}%
     \setunit{\addspace}}%
  \printfield{volume}%
%  \setunit*{\adddot}%
  \printfield{number}%
  \setunit{\addcomma\space}%
  \printfield{eid}%
  \setunit{\addspace}%
  \usebibmacro{issue+date}%
  \newunit\newblock
  \usebibmacro{issue}%
  \newunit}


% Bibliography categories
\def\makebibcategory#1#2{\DeclareBibliographyCategory{#1}\defbibheading{#1}{\section*{#2}}}
\makebibcategory{books}{Books}
\makebibcategory{papers}{Refereed research papers}
\makebibcategory{chapters}{Book chapters}
\makebibcategory{conferences}{Papers in conference proceedings}
\makebibcategory{techreports}{Unpublished working papers}
\makebibcategory{bookreviews}{Book reviews}
\makebibcategory{editorials}{Editorials}
\makebibcategory{phd}{PhD thesis}
\makebibcategory{subpapers}{Submitted papers}
\makebibcategory{curpapers}{Current projects}

\setlength{\bibitemsep}{2.65pt}
\setlength{\bibhang}{.8cm}
\renewcommand{\bibfont}{\small}

\renewcommand*{\bibitem}{\addtocounter{papers}{1}\item \mbox{}\hskip-0.85cm\hbox to 0.85cm{\hfill\arabic{papers}.~~}}
\defbibenvironment{bibliography}
{\list{}
  {\setlength{\leftmargin}{\bibhang}%
   \setlength{\itemsep}{\bibitemsep}%
   \setlength{\parsep}{\bibparsep}}}
{\endlist}
{\bibitem}

\newenvironment{publications}{\section{\LARGE Publications}\label{papersstart}\vspace*{0.2cm}\small
\titlespacing{\section}{0pt}{1.5ex}{1ex}\itemsep=0.00cm
}{\label{papersend}\addtocounter{sumpapers}{-1}\refstepcounter{sumpapers}\label{sumpapers}}

\def\printbib#1{\printbibliography[category=#1,heading=#1]\lastref{sumpapers}}

% Counters for keeping track of papers
\newcounter{papers}\setcounter{papers}{0}
\newcounter{sumpapers}\setcounter{sumpapers}{0}
\def\lastref#1{\addtocounter{#1}{\value{papers}}\setcounter{papers}{0}}

% theorem environments
\declaretheoremstyle[spaceabove=0.25cm,spacebelow=0.25cm,notefont=\normalfont\bfseries, notebraces={(}{)}]{theorem}
\declaretheoremstyle[spaceabove=0.25cm,spacebelow=0.25cm,bodyfont=\normalfont,notefont=\normalfont\bfseries, notebraces={(}{)}]{noital}
\declaretheoremstyle[spaceabove=0.25cm,spacebelow=0.25cm,bodyfont=\normalfont\color{darkgreen},notefont=\normalfont\bfseries, notebraces={(}{)}]{green}
\declaretheoremstyle[spaceabove=0.25cm,spacebelow=0.25cm,bodyfont=\normalfont,notefont=\normalfont\bfseries,qed=$\qedsymbol$,notebraces={(}{)}]{proofstyle}

\declaretheorem[name=Theorem,numberwithin=section,style=theorem]{thm}
\declaretheorem[name=Proposition,sibling=thm,style=theorem]{prop}
\declaretheorem[name=Corollary,sibling=thm,style=theorem]{cor}
\declaretheorem[name=Lemma,sibling=thm,style=theorem]{lem}
\declaretheorem[name=Definition,sibling=thm,style=noital]{defn}
\declaretheorem[name=Example,sibling=thm,style=noital]{example}
\declaretheorem[name=Exercise,numberwithin=section,style=green]{exercise}
\declaretheorem[name=Proof,style=proofstyle,numbered=no]{pf}
\declaretheorem[name=Solution,style=proofstyle,numbered=no]{solution}
\numberwithin{equation}{section}


% macros for convenience
\newcommand{\tops}{\texorpdfstring}

\newcommand{\nid}{\noindent}

\newcommand{\fa}{{\mathfrak a}}
\newcommand{\fp}{{\mathfrak p}}
\newcommand{\fk}{{\mathfrak k}}
\newcommand{\fg}{{\mathfrak g}}
\newcommand{\fh}{{\mathfrak h}}
\newcommand{\fn}{{\mathfrak n}}
\newcommand{\fq}{{\mathfrak q}}
\newcommand{\fm}{{\mathfrak m}}
\newcommand{\fr}{{\mathfrak r}}
\newcommand{\fu}{{\mathfrak u}}
\newcommand{\fG}{{\mathfrak G}}

\newcommand{\cC}{\ensuremath{\mathcal C}}
\newcommand{\cG}{\ensuremath{\mathcal G}}
\newcommand{\cB}{\ensuremath{\mathcal B}}
\newcommand{\cL}{\ensuremath{\mathcal L}}
\newcommand{\cS}{\ensuremath{\mathcal S}}
\newcommand{\cF}{\ensuremath{\mathcal F}}
\newcommand{\cK}{\ensuremath{\mathcal K}}
\newcommand{\cZ}{\ensuremath{\mathcal Z}}
\newcommand{\cM}{\ensuremath{\mathcal M}}
\newcommand{\cN}{\ensuremath{\mathcal N}}
\newcommand{\cO}{\ensuremath{\mathcal O}}
\newcommand{\cH}{\ensuremath{\mathcal H}}
\newcommand{\cX}{\ensuremath{\mathcal X}}
\newcommand{\cY}{\ensuremath{\mathcal Y}}
\newcommand{\cA}{\ensuremath{\mathcal A}}
\newcommand{\cI}{\ensuremath{\mathcal I}}

\newcommand{\R}{\ensuremath{\mathbb R}}
\newcommand{\C}{\ensuremath{\mathbb C}}
\newcommand{\PP}{\ensuremath{\mathbb P}}
\newcommand{\Z}{\ensuremath{\mathbb Z}}
\newcommand{\Q}{\ensuremath{\mathbb Q}}
\newcommand{\A}{\ensuremath{\mathbb A}}
\newcommand{\bbH}{\ensuremath{\mathbb H}}
\newcommand{\bbI}{\ensuremath{\mathbb I}}
\newcommand{\bS}{\ensuremath{\mathbb S}}

\newcommand{\half}{\ensuremath{\frac{1}{2}}}
\newcommand{\qtr}{\ensuremath{\frac{1}{4}}}
\newcommand{\bq}{{\mathbf q}}
\newcommand{\N}{{\mathcal N}}
\newcommand{\F}{{\mathcal F}}
\newcommand{\HH}{{\mathcal H}}
\newcommand{\LL}{{\mathcal L}}
\newcommand{\RR}{{\mathcal R}}
\newcommand{\V}{{\mathcal V}}
\newcommand{\dirac}{\!\!\not\!\partial}
\newcommand{\Dirac}{\!\!\not\!\!D}
\newcommand{\cE}{{\mathcal E}}
\newcommand{\vs}{\not\!v}
\newcommand{\kahler}{K\"ahler\xspace}
\newcommand{\kq}{/\!\!/}
\newcommand{\kql}[1]{/\!\!/\!\!_#1\,}
\newcommand{\hk}{hyperk\"ahler\xspace}
\newcommand{\Hk}{Hyperk\"ahler\xspace}
\newcommand{\hkq}{/\!\!/\!\!/\!\!/}
\newcommand{\hkql}[1]{/\!\!/\!\!/\!\!/\!\!_#1\,}
\newcommand{\del}{\ensuremath{\partial}}
\newcommand{\delbar}{\ensuremath{\overline{\partial}}}
\newcommand{\bl}{{\bf L}}
\newcommand{\J}{{\mathrm j}}
\newcommand{\K}{{\mathrm k}}
\newcommand{\e}{{\mathrm e}}
\newcommand\bid{{\mathbf 1}}
\newcommand{\de}{\mathrm{d}}
\newcommand{\ab}{\mathrm{ab}}
\newcommand{\vol}{\mathrm{vol}}
\renewcommand{\sf}{\mathrm{sf}}
\newcommand{\inst}{\mathrm{inst}}
\newcommand{\eff}{\mathrm{eff}}
\newcommand{\dR}{\mathrm{dR}}
\newcommand{\closed}{\mathrm{closed}}
\newcommand{\exact}{\mathrm{exact}}

\newcommand{\abs}[1]{\lvert#1\rvert}
\newcommand{\norm}[1]{\lVert#1\rVert}
\newcommand{\IP}[1]{\langle#1\rangle}
\newcommand{\DIP}[1]{\langle\!\langle#1\rangle\!\rangle}
\newcommand{\dwrt}[1]{\frac{\partial}{\partial#1}}
\newcommand{\eps}{\epsilon}
\newcommand{\simarrow}{\xrightarrow\sim}

\newcommand{\mmaref}[1]{}

\newcommand{\ti}[1]{\textit{#1}}
\newcommand{\tb}[1]{\textbf{#1}}
\newcommand{\lo}{\text{\calligra o}\,}
\newcommand{\dd}{\ensuremath{\mathscr{D}}}
\newcommand{\bu}{{\bf u}}
\newcommand{\bv}{{\bf v}}
\newcommand{\bw}{{\bf w}}
\newcommand{\bx}{{\bf x}}
\newcommand{\by}{{\bf y}}
\newcommand{\bz}{{\bf z}}
\newcommand{\ba}{{\bf a}}
\newcommand{\bb}{{\bf b}}
\newcommand{\bbr}{{\bf r}}
\newcommand{\bff}{{\bf f}}
\newcommand{\bgg}{{\bf g}}
\newcommand{\bt}{{\bf t}}
\newcommand{\bn}{{\bf n}}
\newcommand{\ii}{{\bf {\hat{i}}}}
\newcommand{\jj}{{\bf {\hat{j}}}}
\newcommand{\kk}{{\bf {\hat{k}}}}


\DeclareMathOperator{\ad}{ad}
\DeclareMathOperator{\im}{Im}
\DeclareMathOperator{\re}{Re}
\DeclareMathOperator{\Tr}{Tr}
\DeclareMathOperator{\End}{End}
\DeclareMathOperator{\Hom}{Hom}
\DeclareMathOperator{\Aut}{Aut}
\DeclareMathOperator{\Sym}{Sym}
\DeclareMathOperator{\Lie}{Lie}
\DeclareMathOperator{\diag}{diag}
\DeclareMathOperator{\Bun}{Bun}
\DeclareMathOperator{\Vect}{Vect}
\DeclareMathOperator{\Span}{Span}
\DeclareMathOperator{\grad}{grad}
\DeclareMathOperator{\rank}{rank}
\DeclareMathOperator{\ind}{ind}
\DeclareMathOperator{\coker}{coker}
\DeclareMathOperator{\Jac}{Jac}
\DeclareMathOperator{\Hol}{Hol}
\DeclareMathOperator{\gr}{gr}

\newcommand{\insfig}[2]{

\medskip
\noindent
\begin{minipage}{\linewidth}

\makebox[\linewidth]{\includegraphics[keepaspectratio=true,scale=#2]{figures/#1-crop.pdf}}

\end{minipage}
\medskip

}


% \newcommand{\insfig}[2]{\begin{figure}[htbp] \centering \includegraphics[scale=#2]{figures/#1-crop.pdf} \label{fig:#1} \end{figure}}
% syntax: \insfig{name}{0.5}{caption}

\newcommand{\fixme}[1]{{\color{orange}{[#1]}}}
\newcommand{\currentposition}{{\color{blue} \noindent\makebox[\linewidth]{\hdashrule{\paperwidth}{1pt}{3mm}}}}

% \mathtoolsset{showonlyrefs}

\bibliography{mvc}

\begin{document}
\pagestyle{fancy}
\lhead{{\tiny \color{gray} \tt \gitAuthorIsoDate}}
\chead{\tiny \ti{Multivariable Calculus, GSMST Spring 2020}}
\rhead{{\tiny \color{gray} \tt \gitAbbrevHash}}
\renewcommand{\headrulewidth}{0.5pt}


\begin{center}
\tb{Multivariable Calculus} \\
Anderson Trimm \\
Gwinnett School of Mathematics, Science and Technology \\
\end{center}

{These are the notes for the Spring Semester 2020
course in Multivariable Calculus at GSMST. They will continually be updated throughout the course. The latest PDF can always be accessed
at \small \url{https://github.com/atrimm/mvc/blob/master/Course%20Notes/multivariable_calculus_2020.pdf.}

\tableofcontents
\renewcommand{\listtheoremname}{Quick reference}
\listoftheorems[onlynamed]

\newpage

%\setcounter{page}{1}

\section{Curves in Spaces}
In this section we study functions with one input and multiple outputs.

\subsection{Vector-valued functions}
\subsubsection{Definitions}
Suppose a particle moves in the plane along the following curve $C$:
	\begin{center}
		\includegraphics[scale=0.5]{figures_mvc/plane_curve}
	\end{center}
Since the curve fails the vertical line test, $C$ cannot be described as the graph of a function $y=f(x)$. Note however that the x- and y-coords of the particle are functions of time
\begin{align*}
	x=f(t), \hspace{0.5cm} y=g(t)
\end{align*}
so the curve $C$ can be described as the image of function ${\bf r}:I \to \mathbb{R}^2$ defined by 
\begin{align*}
	\bbr(t)=(f(t),g(t)),
\end{align*}
where $I=[a,b]$ is an interval in $\R$. \fixme{Add mapping diagram.}

\begin{defn}[Vector-valued function]
Let $U \subseteq \R$. A mapping $\bbr:U \to \mathbb{R}^n$ called a \emph{vector-valued function}. The value of $\bbr$ at $t \in U$ can be written as
\begin{align*}
	\bbr(t)=(r_1(t),r_2(t),\dots,r_n(t))
\end{align*}
where the $n$ functions $r_i:U \to \R$, $i=1,\dots,n$ are called the \emph{component functions} of $\bbr$.	
\end{defn}
Unless specified otherwise, we will take the domain $U$ of a vector-valued function to be the largest domain on which all of the component functions are defined.
\begin{example}
Consider the vector-valued function $\bbr:U \to \mathbb{R}^3$ defined by
	\begin{align*}
		\bbr(t)=(t^3,\ln(3-t),\sqrt{t}).
	\end{align*}
The component functions of $\bbr(t)$ are 
\begin{align*}
	r_1(t)=t^3, \hspace{0.5cm} r_2(t)=\ln(3-t), \hspace{0.5cm} r_3(t)=\sqrt{t}.
\end{align*}
The domains of each of these functions, respectively, are 
\begin{align*}
	U_1=\R, \hspace{0.5cm} U_2=(-\infty,3), \hspace{0.5cm} U_3=[0,\infty),
\end{align*}	
so the domain $U$ of $\bbr(t)$ is 
\begin{align*}
	U=U_1 \cap U_2 \cap U_3=[0,3).
\end{align*} 
\end{example}

\begin{exercise}
	Consider the vector-valued function $\bbr:U \to \mathbb{R}^3$ defined by
	\begin{align*}
		\bbr(t)=\left(\frac{t-2}{t+2},\sin t, \ln(9-t^2)\right).
	\end{align*}
What is the domain $U$ of the function?
\end{exercise}
{\color{red}\begin{solution}
	The component functions of $\bbr(t)$ are 
	\begin{align*}
		r_1=\frac{t-2}{t+2}, \hspace{0.5cm} r_2=\sin t, \hspace{0.5cm} r_3=\ln(9-t^2).
	\end{align*}
	The domains of $r_1(t)$ and $r_2(t)$ are given, respectively, by
	\begin{align*}
		U_1=(-\infty,2) \cup (2,\infty), \hspace{0.5cm} U_2=\R.
	\end{align*}
	To find the domain of $r_3(t)$, we need to solve the inequality
	\begin{align*}
		9-t^2>0. \\
	\end{align*}
	The graph of the function $y=9-x^2$ is a concave-down parabola with $y$-intercept 9 and $x$-intercepts $\pm 3$. 
	
	\begin{figure}[h]
	\begin{center}
		\includegraphics[scale=0.3]{figures_mvc/parab_down}
	\end{center}
	\caption{Graph of $y=9-x^2$.}
\end{figure}
	We have $y>0$ where the graph is above the $x$-axis, so $y>0$ when $-3<x<3$. The domain of $r_3(t)$ is therefore
	\begin{align*}
		U_3=(-3,3).
	\end{align*}
	The domain of $\bbr(t)$ is then
	\begin{align*}
		U=U_1 \cap U_2 \cap U_3=(-3,-2)\cup (-2,3).
	\end{align*}
\end{solution}}

\subsubsection{Review: limits of single-variable functions}
Before considering limits of vector-valued functions, let's review the definition for a real-valued function $y=f(x)$ of a single real variable $x$.

To motivate the definition, consider the function
\begin{align*}
	f(x)=\begin{cases}
		2x-1, \ \text{ if } x \neq 3 \\
		6, \hspace{1.0cm} \text{ if } x = 3 \\
	\end{cases}
\end{align*}
whose graph is shown in the figure below.

\begin{figure}[h]\label{fig:lim_def}
	\begin{center}
		\includegraphics[scale=0.7]{figures_mvc/graph_limit_defn}
	\end{center}
	\caption{Graph of the function $y=f(x)$ in the example above.}
\end{figure}
From the graph, we see that when $x$ is close to 3 but not equal to 3, then $f(x)$ is close to 5, and so $\lim_{x \to 3}f(x)=5$.

To obtain more detailed information about how $f(x)$ varies when $x$ is close to 3, we ask the following question: 

\emph{How close to 3 does $x$ have to be so that $f(x)$ differs from 5 by less than $0.1$?}

The distance from $x$ to $3$ is $|x-3|$ and the distance from $f(x)$ to 5 is $|f(x)-5|$, so our problem is to find a number $\delta$ such that 
\begin{align*}
	|f(x)-5|<0.1 \hspace{0.5cm} \text{ if } \hspace{0.5cm} 0<|x-3|<\delta.
\end{align*}
If $x \neq 3$, then 
\begin{align*}
	|f(x)-5|=|(2x-1)-5|=|2x-6|=2|x-3|
\end{align*}
so we see that by taking $\delta=\frac{1}{2}(0.1)=0.05$, we have $|f(x)-5|<2(0.05)=0.1$. Thus, an answer to the problem is given by $\delta=0.05$; that is, if $x$ is within a distance of $0.05$ from 3, then $f(x)$ will be within a distance of $0.1$ from 5.

If we change the number $0.1$ in our problem to the smaller number $0.01$, then by using the same method we find that $f(x)$ will differ from 5 by less than $0.01$ provided that $x$ differs from 3 by less than $\frac{1}{2}(0.01)=0.005$; that is,
\begin{align*}
	|f(x)-5|<0.01 \hspace{0.5cm} \text{ if } \hspace{0.5cm} 0<|x-3|<0.005.
\end{align*}
Similarly,
\begin{align*}
	|f(x)-5|<0.001 \hspace{0.5cm} \text{ if } \hspace{0.5cm} 0<|x-3|<0.0005.
\end{align*}
Think of the numbers $0.1, 0.01, 0.001$ above as \emph{error tolerances} that we might allow. That is, when challenged with an error tolerance, it is our task to find a corresponding $\delta$ so that whenever $x$ is within a distance of $\delta$ from 3, $f(x)\approx 5$, within the given error tolerance. 

Now for 5 to be the precise limit of $f(x)$ as $x$ approaches 3, we must not only be able to bring the difference between $f(x)$ and 5 below each of these numbers; we must be able to bring it below \emph{any} positive number. And, by exactly the same reasoning, we can. That is, if $\epsilon$ is any positive number, then by choosing $\delta=\frac{\epsilon
}{2}$, we find
\begin{align}\label{eq:review_limit_example}
	|f(x)-5|<\epsilon \hspace{0.5cm} \text{ if } \hspace{0.5cm} 0<|x-3|<\delta=\frac{\epsilon}{2}.
\end{align}
This is a precise way of saying that $f(x)$ is close to 5 when $x$ is close to 3, because Equation \eqref{eq:review_limit_example} says that we can make the values of $f(x)$ within an arbitrary distance $\epsilon$ from 5 by taking the values of $x$ within a distance $\frac{\epsilon}{2}$ from 3 (but $x \neq 3$).

Note that Equation \eqref{eq:review_limit_example} can be rewritten as follows:
\begin{align*}
	\text{ if } \hspace{0.5cm} 3-\delta<x<3+\delta \hspace{0.2cm} (x \neq 3) \hspace{0.5cm} \text{ then } \hspace{0.5cm} 5-\epsilon<f(x)<5+\epsilon
\end{align*}
as illustrated in the figure above. This says that by taking the values of $x$ ($x \neq 3$) to lie in the interval $(3-\delta,3+\delta)$ we can make the values of $f(x)$ lie in the interval $(5-\epsilon,5+\epsilon)$.

Following the reasoning in this example,  the precise definition of a limit is the following.

\begin{defn}[Limit of a single-variable function]\label{def:limit_of_a_single-variable_function}
	Let $(a,b)$ be an open interval containing the point $x_0$ and let $f(x)$ be a real-valued function defined on this interval, except possibly at $x_0$ itself. A number $L$ is called the \emph{limit of $f(x)$ as $x$ approaches $x_0$} if for every $\epsilon>0$ there exists a $\delta>0$ such that $|f(x)-L|<\epsilon$ whenever $0<|x-x_0|<\delta$. If such an $L$ exists, we write
	\begin{align*}
		\lim_{x \to x_0}f(x)=L.
	\end{align*}
\end{defn}

\begin{thm}[Uniqueness of limits]
	If $f(x)$ has a limit $L$ at $x_0$, then the limit is unique.
\end{thm}

\begin{pf}
Suppose that $\lim_{x \to x_0}f(x)=L$ and $\lim_{x \to x_0}f(x)=L'$. Then, given any $\epsilon>0$ there exist positive numbers $\delta_1$ and $\delta_2$ such that 
\begin{align*}
	|f(x)-L|<\frac{\epsilon}{2} \hspace{0.5cm} \text{ if } \hspace{0.5cm} |x-x_0|<\delta_1
\end{align*} 	
and
\begin{align*}
	|f(x)-L'|<\frac{\epsilon}{2} \hspace{0.5cm} \text{ if } \hspace{0.5cm} |x-x_0|<\delta_2.
\end{align*} 
Then by taking $|x-x_0|<\delta=\min\{\delta_1,\delta_2\}$, we have
\begin{align*}
	|L-L'|=|L-L'+f(x)-f(x)|=|L-f(x)+f(x)-L|\leq |f(x)-L|+|f(x)-L'|<\frac{\epsilon}{2}+\frac{\epsilon}{2}=\epsilon.
\end{align*}
For this to be true for all $\epsilon>0$, we must have $L-L'=0$, or $L=L'$.
\end{pf}


\begin{exercise}
Use Definition \ref{def:limit_of_a_single-variable_function} to prove that $\lim_{x \to 3}(4x-5)=7$.
\end{exercise}

{\color{red} \begin{solution}
Let $\epsilon>0$. For all $x \neq 3$, 
\begin{align*}
	|f(x)-7|=|(4x-5)-7|=|4x-12|=4|x-3|.
\end{align*} 
By taking $\delta=\frac{\epsilon}{4}$, we have $0<|x-3|<\frac{\epsilon}{4}$ and therefore
\begin{align*}
	|f(x)-7|=4|x-3|<4\cdot \frac{\epsilon}{4}=\epsilon,
\end{align*}	
which proves that $\lim_{x \to 3}(4x-5)=7$.
\end{solution}}

\begin{example}
We now use Definition \ref{def:limit_of_a_single-variable_function} to prove that $\lim_{x \to 3}x^2=9$.	

Let $\epsilon>0$. For all $x \neq 3$, we have
\begin{align*}
	|f(x)-9|=|x^2-9|=|(x+3)(x-3)|=|x+3||x-3|.
\end{align*} 
Notice that if we can find a positive number $C$ such that $|x+3|<C$, then
\begin{align*}
	|x+3||x-3|<C|x-3|
\end{align*}
and we can make $C|x-3|<\epsilon$ by taking $|x-3|<\frac{\epsilon}{C}=\delta$. We can find such a number $C$ if we restrict $x$ to lie in some interval centered at 3. Since we are only interested in values of $x$ that are close to 3, this is exactly what we want. Let's assume that $|x-3|<\alpha$ for some positive number $\alpha$, say $\alpha=1$ (it does not matter what number we take here). Then we have
\begin{align*}
	x-3<1 \hspace{0.5cm} \text{ or }  \hspace{0.5cm} -x+3<1.
\end{align*}
The first inequality says $x<4$ and the second says $2<x$, so $|x-3|<1$ implies that 
\begin{align*}
	2<x<4.
\end{align*}
Adding 3 to both sides of this inequality gives 
\begin{align*}
	5<x+3<7,
\end{align*}
and therefore $|x+3|<|7|=7=C$. But now there are two restrictions on $|x-3|$, namely
\begin{align*}
	|x-3|<1 \hspace{0.5cm} \text{ and } \hspace{0.5cm} |x-3|<\frac{\epsilon}{C}=\frac{\epsilon}{7}.
\end{align*}
To make sure that both of these inequalities are satisfied, we take $\delta=\min\{1,\frac{\epsilon}{7}\}$. Since $0<|x-3|<\delta$ implies $|x^2-9|<\epsilon$, this proves that $\lim_{x \to 3}x^2=9$.
\end{example}
The previous example shows that it is not always easy to prove that a function has a particular limit using Definition \ref{def:limit_of_a_single-variable_function}. In fact, if we had considered a more complicated function such as 
\begin{align*}
	f(x)=\frac{6x^2-8x+9}{2x^2-1}
\end{align*}
then proving that $\lim_{x \to 1}f(x)=7$ using Definition \ref{def:limit_of_a_single-variable_function} would require a great deal of ingenuity. Instead, we prove the following theorems, which makes evaluating limits much easier.

\begin{lem}[Triangle Inequality]\label{lem:triangle_inequality}
	For all $x,y \in \R$, 
	\begin{align*}
		|x+y| \leq |x|+|y|.
	\end{align*}
\end{lem}

\begin{pf}
We have
\begin{align*}
	|x+y|^2=(x+y)^2&=x^2+y^2+2xy \\
	&=|x|^2+|y|^2+2xy \\
	&\leq |x|^2+|y|^2+2|x||y| \\
	&=(|x|+|y|)^2.
\end{align*}	
Since both sides are nonnegative, this implies that 
\begin{align*}
	|x+y| \leq |x|+|y|.
\end{align*}
\end{pf}



\begin{thm}[Limit laws for single-variable functions]\label{thm:limit_laws_for_single-variable_functions}
Suppose $f(x)$ and $g(x)$ are defined on the same open set containing $x_0$, and that 
\begin{align*}
	\lim_{x \to x_0}f(x)=L \hspace{0.5cm} \text{ and } \hspace{0.5cm} \lim_{x \to x_0}g(x)=M.
\end{align*}
Then
	\begin{enumerate}[(i)]
		\item $\lim_{x \to x_0}c=c$ for any constant $c \in \R$.
		\item $\lim_{x \to x_0}x=x_0$.
		\item $\lim_{x \to x_0} cf(x)=cL$ for any $c \in \R$;
		\item $\lim_{x \to x_0}(f(x)+ g(x))=L+M$;
		\item $\lim_{x \to x_0}(f(x)g(x))=LM$;
		\item $\lim_{x \to x_0}=\frac{f(x)}{g(x)}=\frac{L}{M}$ whenever $M \neq 0$.
	\end{enumerate}
\end{thm}

\begin{pf}
\begin{enumerate}[(i)]
	\item Let $\epsilon>0$. Since $|c-c|=0$, $|c-c|<\epsilon$ whenever $|x-x_0|<\delta$ for any positive number $\delta$.
	\item Given $\epsilon>0$, by taking $\delta=\epsilon$ we have $|x-x_0|<\epsilon$ whenever $|x-x_0|<\delta=\epsilon$.
	\item Since $\lim_{x \to x_0}f(x)=L$, given $\epsilon>0$ there exists a corresponding $\delta>0$ such that $|f(x)-L|<\epsilon$ whenever $0<|x-x_0|<\delta$. Then $|cf(x)-cL|=|c||f(x)-L|<\epsilon$ whenever $0<|x-x_0|<\frac{\epsilon}{|c|}$.
	\item We have
	\begin{align*}
		|f(x)+g(x)-(L+M)|=|(f(x)-L)+(g(x)-M)| \leq |f(x)-L|+|g(x)-M|
	\end{align*}
	by the Triangle Inequality (Lemma \ref{lem:triangle_inequality}). Since $\lim_{x \to x_0}f(x)=L$ and $\lim_{x \to x_0}g(x)=M$, given $\epsilon>0$ there exist positive numbers $\delta_1$ and $\delta_2$ such that 
	\begin{align*}
		|f(x)-L|<\frac{\epsilon}{2} \hspace{0.5cm} \text{ if } \hspace{0.5cm} |x-x_0|<\delta_1
	\end{align*} 
	and 
	\begin{align*}
		|g(x)-M|<\frac{\epsilon}{2} \hspace{0.5cm} \text{ if } \hspace{0.5cm} |x-x_0|<\delta_2.
	\end{align*} 
	By taking $|x-x_0|<\delta=\min\{\delta_1,\delta_2\}$, we have 
	\begin{align*}
		|f(x)+g(x)-(L+M)|\leq |f(x)-L|+|g(x)-M|<\frac{\epsilon}{2}+\frac{\epsilon}{2}=\epsilon,
	\end{align*}
	which proves that $\lim_{x \to x_0}(f(x)+g(x))=L+M$.
	\item First, note that
		\begin{align*}
			f(x)g(x)-LM&=(f(x)-L)(g(x)-M)+L(g(x)-M)+M(f(x)-L).
		\end{align*} 
		Let $\epsilon>0$. Since $\lim_{x \to x_0}f(x)=L$ there exists $\delta_1>0$ such that $|f(x)-L|<\sqrt{\epsilon}$ whenever $|x-x_0|<\delta_1$. Since $\lim_{x \to x_0}g(x)=M$ there exists $\delta_2>0$ such that $|g(x)-M|<\sqrt{\epsilon}$ whenever $|x-x_0|<\delta_2$. Then, whenever $|x-x_0|<\delta=\min\{\delta_1,\delta_2\}$, we have
		\begin{align*}
			|(f(x)-L)(g(x)-M)|=|f(x)-L||g(x)-M|<(\sqrt{\epsilon})^2=\epsilon
		\end{align*}
		which shows that $\lim_{x-x_0}(f(x)-L)(g(x)-M)=0$. By (iii), 
		\begin{align*}
			\lim_{x \to x_0}L(g(x)-M)=L\lim_{x \to x_0}(g(x)-M)=L\cdot0=0,
		\end{align*}
		 and 
		\begin{align*}
			\lim_{x \to x_0}M(f(x)-L)=M\lim_{x \to x_0}(f(x)-L)=M\cdot0=0.
		\end{align*}
		Applying (iv),
		\begin{align*}
			\lim_{x \to x_0}(f(x)g(x)-LM)&=\lim_{x \to x_0}(f(x)-L)(g(x)-M)+\lim_{x \to x_0}L(g(x)-M)+\lim_{x \to x_0}M(f(x)-L) \\
			&=0+0+0 \\
			&=0,
		\end{align*}
		and therefore
		\begin{align*}
			\lim_{x \to x_0}f(x)g(x)=LM.
		\end{align*}
	\item First, note that since $|M|>0$ and $\lim_{x \to x_0}g(x)=M$, there exists $\delta_1>0$ such that $|g(x)|>\frac{1}{2}|M|$ whenever $|x-x_0|<\delta_1$ \fixme{Draw a picture.}. Let $\epsilon>0$. Choose $\delta_2>0$ such that $|x-x_0|<\delta_2$ implies that $|g(x)-M|<\frac{1}{2}|M|^2\epsilon$. Then, for $|x-x_0|<\delta = \min\{\delta_1,\delta_2\}$, we have
	\begin{align*}
		|\frac{1}{g(x)}-\frac{1}{M}|&=|\frac{M-g(x)}{Mg(x)}|\\
		&=\frac{|g(x)-M|}{|Mg(x)|} \\
		&<\frac{\frac{1}{2}|M|^2\epsilon}{\frac{1}{2}|M|^2} \\
		&=\epsilon,	
	\end{align*}
	and therefore $\lim_{x \to x_0}\frac{1}{g(x)}=\frac{1}{M}$. It then follows from (v) that
	\begin{align*}
		\lim_{x \to x_0}\frac{f(x)}{g(x)}&=\lim_{x \to x_0}f(x) \lim_{x \to x_0}\frac{1}{g(x)} \\
		&=\frac{L}{M}.
	\end{align*}
\end{enumerate}	
\end{pf}
Using Theorem \ref{thm:limit_laws_for_single-variable_functions}, it is much easier to prove the limits in the examples above. For instance
\begin{align*}
	\lim_{x \to 3}(4x-5)&=(\lim_{x \to 3}4)(\lim_{x \to 3}x)+(\lim_{x \to 3}(-5)) \\
	&=4(3)+(-5) \\
	&=12-5 \\
	&=7,
\end{align*}
and 
\begin{align*}
	\lim_{x \to 3}x^2=(\lim_{x \to 3}x)(\lim_{x \to 3}x)=(3)(3)=9.
\end{align*}
Note that, in both of these examples, the function $f(x)$ is actually defined at $x_0$ and $\lim_{x \to x_0}f(x)=f(x_0)$; that is, the limit of $f(x)$ as $x$ approaches $x_0$ is equal to the value of $f(x)$ at $x_0$.

\begin{defn}[Continuity]
	Let $f(x)$ be defined on an open interval $(a,b)$ containing a point $x_0$. We say that $f(x)$ is \emph{continuous at $x_0$} if $\lim_{x \to x_0}f(x)=f(x_0)$. We then say that $f(x)$ is \emph{continuous on $(a,b)$} if $f(x)$ is continuous at every point in $(a,b)$.
\end{defn}
The limit laws in Theorem \ref{thm:limit_laws_for_single-variable_functions} imply that
\begin{itemize}
	\item Polynomials are continuous on $\R$;
	\item Rational functions are continuous wherever they are defined;
	\item The absolute value function $f(x)=|x|$ is continuous;
\end{itemize}
Trig functions, and exponential and logarithmic functions are all also continuous wherever they are defined.

\begin{exercise}
Prove that $f(x)=|x|$ is continuous on $\R$.	
\end{exercise}

{\color{red} \begin{solution}
 	If $x>0$, then $f(x)=x$ which is continuous since it is a polynomial. The same is true for $x<0$ since then $f(x)=-x$. By taking $\delta=\epsilon$, $|f(x)-0|=||x||=|x|<\epsilon$ whenever $|x|<\delta=\epsilon$, so $\lim_{x \to 0}f(x)=0=f(0)$, which shows that $f(x)$ is also continuous at $x=0$. Thus, $f(x)$ is continuous on $\R$.
 \end{solution}}
The following theorem is also very useful.

\begin{thm}[A composition of continuous functions is continuous]\label{thm:a_composition_of_continuous_functions_is_continuous}
	Suppose $f(x)$ is defined on an open interval containing $x_0$ and  $g(x)$ is defined on an open interval containing $f(x_0)$. If $f$ is continuous at $x_0$ and $g(x)$ is continuous at $f(x_0)$, then $(g \circ f)(x)$ is continuous at $x_0$.
\end{thm}

\begin{pf}
Let $\epsilon>0$. Since $g$ is continuous at $f(x_0)$, corresponding to $\epsilon$ there exists $\eta>0$ such that $|g(f(x))-g(f(x_0))|<\epsilon$ whenever $|f(x)-f(x_0)|<\eta$. Since $f$ is continuous at $x_0$, corresponding to $\eta$ there exists $\delta>0$ such that $|f(x)-f(x_0)|<\eta$ whenever $|x-x_0|<\delta$. This shows that $|g(f(x))-g(f(x_0))|<\epsilon$ whenever $|x-x_0|<\delta$, proving that $(g \circ f)(x)$ is continuous at $x_0$.
\end{pf}

\begin{example}
Consider the function $f(x)=e^{x^2}$. We can view $f(x)$ as the composition $(h \circ g)(x)$, where $h(x)=e^x$ and $g(x)=x^2$. Since $h(x)$ and $g(x)$ are continuous on $\R$, by Theorem \ref{thm:a_composition_of_continuous_functions_is_continuous} so is $f(x)$.
\end{example}

\subsubsection{Limits of vector-valued functions}
\begin{defn}[Limit of a vector-valued function]
	Let $I=(a,b)$ be an open interval in $\R$ containing $t_0$, let $\bbr:I\to \mathbb{R}^3$ be a vector valued function defined on $I$, except perhaps at $t_0$ itself, and let $\bl$ be a fixed vector in $\mathbb{R}^n$. We say that $\bl$ $\lim_{t \to t_0}\bbr(t)={\bf L}$ if for every $\epsilon >0$ there exists a corresponding $\delta > 0$ such that
	\begin{align*}
		0 < |t-t_0|<\delta \implies ||\bbr(t)-{\bf L}||<\epsilon.
	\end{align*}
\end{defn}

\end{document}

