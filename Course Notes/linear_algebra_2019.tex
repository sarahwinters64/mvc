\documentclass[12pt,letterpaper,reqno]{article}

% \usepackage{mathtools}
\usepackage{epsfig}
\usepackage{amsmath}
\usepackage{amssymb}
\usepackage{amsthm}
\usepackage{indentfirst}
\usepackage{xspace}
\usepackage{multirow}
\usepackage{hyperref}
\usepackage{xcolor}
\usepackage{verbatim}
\usepackage[letterpaper,margin=1in,headheight=15pt]{geometry}
\usepackage{mathpazo}
\usepackage{tikz-cd}
\usepackage{booktabs}
\usepackage{framed}
\usepackage{float}
\usepackage{thmtools}
\usepackage{dashrule}
\usepackage[missing=]{gitinfo2}
\usepackage{fancyhdr}
\usepackage{enumerate}
\usepackage{graphicx}
\usepackage{mathrsfs}
\usepackage{calligra}
\usepackage[titletoc,title]{appendix}
\usepackage{tikz}
\usetikzlibrary{decorations.markings}
\usetikzlibrary{arrows}

\definecolor{darkblue}{rgb}{0.1,0.1,0.7}
\definecolor{darkred}{rgb}{0.5,0.1,0.1}
\definecolor{darkgreen}{rgb}{0.0,0.42,0.06}
\hypersetup{colorlinks=true,urlcolor=darkred,linkcolor=darkblue,citecolor=darkred}
\definecolor{shadecolor}{rgb}{0.85,0.85,0.85}

% Bibliography formatting
\usepackage[bibstyle=authoryear-comp,labeldate=false,defernumbers=true,maxnames=20,uniquename=init,dashed=false,backend=biber,sorting=none]{biblatex}

\DeclareNameAlias{sortname}{first-last}

\DeclareFieldFormat{url}{\url{#1}}
\DeclareFieldFormat[article]{pages}{#1}
\DeclareFieldFormat[inproceedings]{pages}{\lowercase{pp.}#1}
\DeclareFieldFormat[incollection]{pages}{\lowercase{pp.}#1}
\DeclareFieldFormat[article]{volume}{\textbf{#1}}
\DeclareFieldFormat[article]{number}{(#1)}
\DeclareFieldFormat[article]{title}{\MakeCapital{#1}}
\DeclareFieldFormat[inproceedings]{title}{#1}
\DeclareFieldFormat{shorthandwidth}{#1}

% Don't use "In:" in bibliography. Omit urls from journal articles.
\DeclareBibliographyDriver{article}{%
  \usebibmacro{bibindex}%
  \usebibmacro{begentry}%
  \usebibmacro{author/editor}%
  \setunit{\labelnamepunct}\newblock
  \MakeSentenceCase{\usebibmacro{title}}%
  \newunit
  \printlist{language}%
  \newunit\newblock
  \usebibmacro{byauthor}%
  \newunit\newblock
  \usebibmacro{byeditor+others}%
  \newunit\newblock
  \printfield{version}%
  \newunit\newblock
%  \usebibmacro{in:}%
  \usebibmacro{journal+issuetitle}%
  \newunit\newblock
  \printfield{note}%
  \setunit{\bibpagespunct}%
  \printfield{pages}
  \newunit\newblock
  \usebibmacro{eprint}
  \newunit\newblock
  \printfield{addendum}%
  \newunit\newblock
  \usebibmacro{pageref}%
  \usebibmacro{finentry}}

% Remove dot between volume and number in journal articles.
\renewbibmacro*{journal+issuetitle}{%
  \usebibmacro{journal}%
  \setunit*{\addspace}%
  \iffieldundef{series}
    {}
    {\newunit
     \printfield{series}%
     \setunit{\addspace}}%
  \printfield{volume}%
%  \setunit*{\adddot}%
  \printfield{number}%
  \setunit{\addcomma\space}%
  \printfield{eid}%
  \setunit{\addspace}%
  \usebibmacro{issue+date}%
  \newunit\newblock
  \usebibmacro{issue}%
  \newunit}


% Bibliography categories
\def\makebibcategory#1#2{\DeclareBibliographyCategory{#1}\defbibheading{#1}{\section*{#2}}}
\makebibcategory{books}{Books}
\makebibcategory{papers}{Refereed research papers}
\makebibcategory{chapters}{Book chapters}
\makebibcategory{conferences}{Papers in conference proceedings}
\makebibcategory{techreports}{Unpublished working papers}
\makebibcategory{bookreviews}{Book reviews}
\makebibcategory{editorials}{Editorials}
\makebibcategory{phd}{PhD thesis}
\makebibcategory{subpapers}{Submitted papers}
\makebibcategory{curpapers}{Current projects}

\setlength{\bibitemsep}{2.65pt}
\setlength{\bibhang}{.8cm}
\renewcommand{\bibfont}{\small}

\renewcommand*{\bibitem}{\addtocounter{papers}{1}\item \mbox{}\hskip-0.85cm\hbox to 0.85cm{\hfill\arabic{papers}.~~}}
\defbibenvironment{bibliography}
{\list{}
  {\setlength{\leftmargin}{\bibhang}%
   \setlength{\itemsep}{\bibitemsep}%
   \setlength{\parsep}{\bibparsep}}}
{\endlist}
{\bibitem}

\newenvironment{publications}{\section{\LARGE Publications}\label{papersstart}\vspace*{0.2cm}\small
\titlespacing{\section}{0pt}{1.5ex}{1ex}\itemsep=0.00cm
}{\label{papersend}\addtocounter{sumpapers}{-1}\refstepcounter{sumpapers}\label{sumpapers}}

\def\printbib#1{\printbibliography[category=#1,heading=#1]\lastref{sumpapers}}

% Counters for keeping track of papers
\newcounter{papers}\setcounter{papers}{0}
\newcounter{sumpapers}\setcounter{sumpapers}{0}
\def\lastref#1{\addtocounter{#1}{\value{papers}}\setcounter{papers}{0}}

% theorem environments
\declaretheoremstyle[spaceabove=0.25cm,spacebelow=0.25cm,notefont=\normalfont\bfseries, notebraces={(}{)}]{theorem}
\declaretheoremstyle[spaceabove=0.25cm,spacebelow=0.25cm,bodyfont=\normalfont,notefont=\normalfont\bfseries, notebraces={(}{)}]{noital}
\declaretheoremstyle[spaceabove=0.25cm,spacebelow=0.25cm,bodyfont=\normalfont\color{darkgreen},notefont=\normalfont\bfseries, notebraces={(}{)}]{green}
\declaretheoremstyle[spaceabove=0.25cm,spacebelow=0.25cm,bodyfont=\normalfont,notefont=\normalfont\bfseries,qed=$\qedsymbol$,notebraces={(}{)}]{proofstyle}

\declaretheorem[name=Theorem,numberwithin=section,style=theorem]{thm}
\declaretheorem[name=Proposition,sibling=thm,style=theorem]{prop}
\declaretheorem[name=Corollary,sibling=thm,style=theorem]{cor}
\declaretheorem[name=Lemma,sibling=thm,style=theorem]{lem}
\declaretheorem[name=Definition,sibling=thm,style=noital]{defn}
\declaretheorem[name=Example,sibling=thm,style=noital]{example}
\declaretheorem[name=Exercise,numberwithin=section,style=green]{exercise}
\declaretheorem[name=Proof,style=proofstyle,numbered=no]{pf}

\numberwithin{equation}{section}


% macros for convenience
\newcommand{\tops}{\texorpdfstring}

\newcommand{\nid}{\noindent}

\newcommand{\fa}{{\mathfrak a}}
\newcommand{\fp}{{\mathfrak p}}
\newcommand{\fk}{{\mathfrak k}}
\newcommand{\fg}{{\mathfrak g}}
\newcommand{\fh}{{\mathfrak h}}
\newcommand{\fn}{{\mathfrak n}}
\newcommand{\fq}{{\mathfrak q}}
\newcommand{\fm}{{\mathfrak m}}
\newcommand{\fr}{{\mathfrak r}}
\newcommand{\fu}{{\mathfrak u}}
\newcommand{\fG}{{\mathfrak G}}

\newcommand{\cC}{\ensuremath{\mathcal C}}
\newcommand{\cG}{\ensuremath{\mathcal G}}
\newcommand{\cB}{\ensuremath{\mathcal B}}
\newcommand{\cL}{\ensuremath{\mathcal L}}
\newcommand{\cS}{\ensuremath{\mathcal S}}
\newcommand{\cF}{\ensuremath{\mathcal F}}
\newcommand{\cK}{\ensuremath{\mathcal K}}
\newcommand{\cZ}{\ensuremath{\mathcal Z}}
\newcommand{\cM}{\ensuremath{\mathcal M}}
\newcommand{\cN}{\ensuremath{\mathcal N}}
\newcommand{\cO}{\ensuremath{\mathcal O}}
\newcommand{\cH}{\ensuremath{\mathcal H}}
\newcommand{\cX}{\ensuremath{\mathcal X}}
\newcommand{\cY}{\ensuremath{\mathcal Y}}
\newcommand{\cA}{\ensuremath{\mathcal A}}
\newcommand{\cI}{\ensuremath{\mathcal I}}

\newcommand{\R}{\ensuremath{\mathbb R}}
\newcommand{\C}{\ensuremath{\mathbb C}}
\newcommand{\PP}{\ensuremath{\mathbb P}}
\newcommand{\Z}{\ensuremath{\mathbb Z}}
\newcommand{\Q}{\ensuremath{\mathbb Q}}
\newcommand{\A}{\ensuremath{\mathbb A}}
\newcommand{\bbH}{\ensuremath{\mathbb H}}
\newcommand{\bbI}{\ensuremath{\mathbb I}}
\newcommand{\bS}{\ensuremath{\mathbb S}}

\newcommand{\half}{\ensuremath{\frac{1}{2}}}
\newcommand{\qtr}{\ensuremath{\frac{1}{4}}}
\newcommand{\bq}{{\mathbf q}}
\newcommand{\N}{{\mathcal N}}
\newcommand{\F}{{\mathcal F}}
\newcommand{\HH}{{\mathcal H}}
\newcommand{\LL}{{\mathcal L}}
\newcommand{\RR}{{\mathcal R}}
\newcommand{\V}{{\mathcal V}}
\newcommand{\dirac}{\!\!\not\!\partial}
\newcommand{\Dirac}{\!\!\not\!\!D}
\newcommand{\cE}{{\mathcal E}}
\newcommand{\vs}{\not\!v}
\newcommand{\kahler}{K\"ahler\xspace}
\newcommand{\kq}{/\!\!/}
\newcommand{\kql}[1]{/\!\!/\!\!_#1\,}
\newcommand{\hk}{hyperk\"ahler\xspace}
\newcommand{\Hk}{Hyperk\"ahler\xspace}
\newcommand{\hkq}{/\!\!/\!\!/\!\!/}
\newcommand{\hkql}[1]{/\!\!/\!\!/\!\!/\!\!_#1\,}
\newcommand{\del}{\ensuremath{\partial}}
\newcommand{\delbar}{\ensuremath{\overline{\partial}}}
\newcommand{\I}{{\mathrm i}}
\newcommand{\J}{{\mathrm j}}
\newcommand{\K}{{\mathrm k}}
\newcommand{\e}{{\mathrm e}}
\newcommand\bid{{\mathbf 1}}
\newcommand{\de}{\mathrm{d}}
\newcommand{\ab}{\mathrm{ab}}
\newcommand{\vol}{\mathrm{vol}}
\renewcommand{\sf}{\mathrm{sf}}
\newcommand{\inst}{\mathrm{inst}}
\newcommand{\eff}{\mathrm{eff}}
\newcommand{\dR}{\mathrm{dR}}
\newcommand{\closed}{\mathrm{closed}}
\newcommand{\exact}{\mathrm{exact}}

\newcommand{\abs}[1]{\lvert#1\rvert}
\newcommand{\norm}[1]{\lVert#1\rVert}
\newcommand{\IP}[1]{\langle#1\rangle}
\newcommand{\DIP}[1]{\langle\!\langle#1\rangle\!\rangle}
\newcommand{\dwrt}[1]{\frac{\partial}{\partial#1}}
\newcommand{\eps}{\epsilon}
\newcommand{\simarrow}{\xrightarrow\sim}

\newcommand{\mmaref}[1]{}

\newcommand{\ti}[1]{\textit{#1}}
\newcommand{\tb}[1]{\textbf{#1}}
\newcommand{\lo}{\text{\calligra o}\,}
\newcommand{\dd}{\ensuremath{\mathscr{D}}}
\newcommand{\sgn}{\text{sgn}}


\DeclareMathOperator{\ad}{ad}
\DeclareMathOperator{\im}{Im}
\DeclareMathOperator{\re}{Re}
\DeclareMathOperator{\Tr}{Tr}
\DeclareMathOperator{\End}{End}
\DeclareMathOperator{\Hom}{Hom}
\DeclareMathOperator{\Aut}{Aut}
\DeclareMathOperator{\Sym}{Sym}
\DeclareMathOperator{\Lie}{Lie}
\DeclareMathOperator{\diag}{diag}
\DeclareMathOperator{\Bun}{Bun}
\DeclareMathOperator{\Vect}{Vect}
\DeclareMathOperator{\Span}{Span}
\DeclareMathOperator{\grad}{grad}
\DeclareMathOperator{\rank}{rank}
\DeclareMathOperator{\ind}{ind}
\DeclareMathOperator{\coker}{coker}
\DeclareMathOperator{\Jac}{Jac}
\DeclareMathOperator{\Hol}{Hol}
\DeclareMathOperator{\gr}{gr}

\newcommand{\insfig}[2]{

\medskip
\noindent
\begin{minipage}{\linewidth}

\makebox[\linewidth]{\includegraphics[keepaspectratio=true,scale=#2]{figures/#1-crop.pdf}}

\end{minipage}
\medskip

}


% \newcommand{\insfig}[2]{\begin{figure}[htbp] \centering \includegraphics[scale=#2]{figures/#1-crop.pdf} \label{fig:#1} \end{figure}}
% syntax: \insfig{name}{0.5}{caption}

\newcommand{\fixme}[1]{{\color{orange}{[#1]}}}
\newcommand{\currentposition}{{\color{blue} \noindent\makebox[\linewidth]{\hdashrule{\paperwidth}{1pt}{3mm}}}}

% \mathtoolsset{showonlyrefs}

\bibliography{mvc}

\begin{document}
\pagestyle{fancy}
\lhead{{\tiny \color{gray} \tt \gitAuthorIsoDate}}
\chead{\tiny \ti{Linear Algebra, GSMST 2018-2019}}
\rhead{{\tiny \color{gray} \tt \gitAbbrevHash}}
\renewcommand{\headrulewidth}{0.5pt}


\begin{center}
\tb{Multivariable Calculus \\
Semester 1: Linear Algebra} \\
Anderson Trimm \\
Gwinnett School of Mathematics, Science and Technology \\
\end{center}

{These are the notes for the Fall Semester 2019
of Multivariable Calculus at GSMST, which covers linear algebra. They will updated frequently throughout the semester. The latest PDF can always be accessed
at \small \url{https://github.com/atrimm/mvc/blob/master/Course%20Notes/linear_algebra_2019.pdf}.} Please email me with comments and corrections, or send them to me directly as pull requests to the source repository hosted at \small \url{https://github.com/atrimm/mvc}.

\tableofcontents
\renewcommand{\listtheoremname}{Quick reference}
\listoftheorems[onlynamed]

\newpage

%\setcounter{page}{1}
\section{Vectors and geometry}
\subsection{Physical motivation}
The earliest notion of a \emph{vector} comes from physics. In nature, we encounter certain physical quantities which cannot be uniquely specified by a number alone, but also depend on a direction in space. 

\begin{example}\label{ex:disp}
If the distance from town $A$ to town $B$ is 400 miles and we leave $A$ and travel at 50 miles per hour, then we will arrive at $B$ in 8 hours, but only if we travel in the direction from $A$ to $B$! Thus, displacement (400 mi, from $A$ to $B$) and velocity (50 mi/hr, from $A$ to $B$) are two examples of such physical quantities.
\end{example}
To distinguish physical quantities which depend on a numerical value alone from those which also depend on a direction, we make the following definitions. 
\begin{defn}[Vectors and scalars]
\begin{enumerate}[(a)] \hspace{10cm}
	\item A \emph{scalar} is a physical quantity which is uniquely specified by a numerical value alone.
	\item A \emph{vector} is a physical quantity which is uniquely specified by a numerical value, called its \emph{magnitude}, and a direction.
\end{enumerate}
\end{defn}

\begin{exercise}
Classify each of the following quantities are vector or scalar:
\begin{enumerate}[(a)]
	\item Force
	\item Temperature
	\item Mass
	\item Volume
	\item Acceleration
	\item Electric Charge
	\item Density
\end{enumerate}	
\end{exercise}

In the following sections, we will develop a mathematical model of vector and scalar quantities capable of modeling physical phenomena. As we will see, this model will have applications beyond physics as well.

\subsection{Mathematical model}
\subsubsection{Scalars}\label{sec:scalars}
While electric charges are observed in nature to take only integer values, other scalar quantities, such as mass and temperature, are found to take any \emph{real} value. Thus, in our model a scalar is simply a real number. We denote the set of all real numbers by $\mathbb{R}$. We will denote a scalar by a lower case latin letter, such as $x$.

\subsubsection{Vectors}
A vector quantity is uniquely specified by two pieces of information: a magnitude (which is a nonnegative real number) and a direction in three-dimensional space. We can therefore represent a vector geometrically as an \emph{arrow} (directed line segment) in space.  The arrow points in the direction specifying the vector while the length of the arrow represents the magnitude of the vector. For example, a force of 25 N directed at an angle of $45^\circ$ with respect to the positive $x$-axis is represented by the arrow \fixme{Insert graphic.} We will denote a vector by a boldface latin letter (either upper or lowercase), such as $\mathbf{x}$. When writing by hand, a more common notation is $\vec{x}$.

\subsubsection{Equality of vectors}
We will now discuss the notion of equivalence of two vectors. Recall first the equivalence of plane figures, say triangles. We consider two distinct triangles. Since the data defining a triangle are the side lengths and interior angle measures, any two triangles related by a transformation which leaves these unchanged represents an equivalent triangle; the only difference between them is the location in the plane. As you learned in basic algebra, any transformation of the plane which preserves lengths and angles can be written as a finite sequence of reflections, translations, and rotations and any two triangles related by such a transformation are said to be \emph{congruent}. 

Since the defining data of a vector is the magnitude and direction, we will agree that

\begin{defn}[Equality of vectors]
Two vectors are equal if they have the same magnitude and direction.	
\end{defn}
That is, two vectors are equal if they are represented by parallel line segments which have the same length and orientation. 

\begin{center}
	\includegraphics[scale=0.5]{figures_mvc/equal_vectors_new}
\end{center}
If two vectors $\mathbf{x}$ and $\mathbf{y}$ are equal, we will denote this by $\mathbf{x}=\mathbf{y}$.

It is important to note that we have defined equality of vectors so that \emph{location} in space does \emph{not} matter. As line segments, the two parallel arrows are congruent, \footnote{Since they have the same length; the fact that they are parallel or have the same orientation does not matter.} but still considered distinct due to their difference in location, but as \emph{vectors} they are regarded as exactly the same vector.

\subsection{Vector arithmetic}
We will now define operations involving vectors, whose motivation will come from physics. \footnote{In defining operations on the set of vectors, it is important that we have explicitly stated when two vectors are to be regarded being equal, as this will be needed to ensure that our definitions make sense. For instance, consider the definition of addition of rational numbers:
\begin{align}\label{eq:sum_good}
	\frac{m}{n}+\frac{p}{q}=\frac{mq+np}{nq}.
\end{align}
Suppose instead we try to define addition by 
\begin{align}\label{eq:sum_bad}
	\frac{m}{n}+\frac{p}{q}=\frac{m+p}{n+q}.
\end{align}
With this definition, we have, for example,
	$\frac{1}{2}+\frac{1}{3}=\frac{1+1}{2+3}=\frac{2}{5}$.
Now, in both these definitions, the formula for the sum has a \emph{massive} built-in ambiguity: the symbol ``$\frac{1}{2}$" is just one representative of the infinite set $\{\frac{1}{2}, \frac{2}{4}, \frac{3}{6},\dots\}$. Therefore the sum
	$\frac{1}{2}+\frac{1}{3}$
should be equivalent to 
	$\frac{2}{4}+\frac{3}{9}$.
However, with definition \eqref{eq:sum_bad}, we have
$\frac{1}{2}+\frac{1}{3}=\frac{2}{5}$
while
$\frac{2}{4}+\frac{3}{9}=\frac{5}{13}$
and $\frac{2}{5} \neq \frac{5}{13}$. Thus definition \eqref{eq:sum_bad} makes \emph{absolutely no sense}.

On the other hand, using the proper definition \eqref{eq:sum_good}, we have
	$\frac{1}{2}+\frac{1}{3}=\frac{5}{6}$
while
$\frac{2}{4}+\frac{3}{9}=\frac{30}{36}$ 
and indeed $\frac{5}{6}=\frac{30}{36}$ as rational numbers.

To see what went wrong for \eqref{eq:sum_bad}, note that the operation in both \eqref{eq:sum_bad} and \eqref{eq:sum_good} required \emph{choosing} one of the elements of the sets $\{\frac{1}{2},\frac{2}{4},\dots\}$ and $\{\frac{1}{3},\frac{3}{9},\dots\}$ and applying the formula on the right hand side. For the definition to make sense, the result of the operation must be \emph{independent} of this choice. We describe this issue by saying that we have to verify that our proposed operation is \emph{well-defined}.}

\subsubsection{Vector addition}
The most basic operation one can define on a set is a \emph{binary operation}, which is a rule for combining any two elements in the set to produce a third element in the set. \footnote{More formally, we write this as a map $V \times V \to V$, where $V \times V=\{(v,w) \mid v,w \in V\}$ denotes the set of all ordered pairs of elements of $V$, called the \emph{Cartesian product} of $V$ with itself.} There is a natural binary operation on the set of vectors, which is suggested by the \emph{force table} experiment in mechanics.

\begin{center}
	\includegraphics[scale=0.5]{figures_mvc/force_table_experiment_2}
\end{center}

In a force table experiment, strings are tied to a metal ring which is positioned at the center of the table. The strings are then suspended over pulleys which are fixed at known angles, and known masses hung from the ends of the strings. The pull of gravity on a given mass creates tension in the string which pulls on the ring. 

In an experiment in which \emph{two} strings are tied to the ring, the tension in each string gives rise to two forces pulling on the washer in different directions. However, the washer ultimately accelerates in a single direction, which is the direction of the \emph{net} (or \emph{total}) force acting on the ring. The rule for combining the two tension force vectors to produce the net force vector is exactly the binary operation we seek to define. 

To determine the net force, a third string is connected to the ring with mass and pulley position chosen so that the ring is in \emph{static equilibrium} (i.e., it does not move at all under the influence of these three forces). This vector is called the \emph{equilibrant} vector. By Newton's third law, the net force vector (also called the \emph{resultant} vector) is then the \emph{opposite} of the equilibrant vector, that is, it has the same magnitude and is directed along the same line, but with the opposite orientation.

\begin{center}
	\includegraphics[scale=0.5]{figures_mvc/force-table-equi-res}
\end{center}

 We therefore define the \emph{sum} of two vectors as follows:
 
 \begin{defn}[Vector addition]
 	The \ti{sum} ${\bf v}+{\bf w}$ of two vectors ${\bf v}$ and ${\bf w}$ is the resultant vector of ${\bf v}$ and ${\bf w}$.
 \end{defn}
Note that when the two tension forces are along the \emph{same} direction (e.g., just add another mass on the same string), the resultant vector points in this same direction and has magnitude given by the sum of the magnitudes of the two tension vectors, and hence the addition of ${\bf v}$ and ${\bf w}$ reduces to addition of ordinary numbers in this special case. This is why we have decided to call this binary operation \emph{addition} and to continue to denote it by $+$; it can therefore be thought of as a \emph{generalization} of the ordinary addition of scalars. 
 
In terms of our geometric representation of vectors, the magnitude and direction of ${\bf v}+{\bf w}$ is determined as follows: Since the location of a vector is of no consequence (by our definition of equality of vectors), we may position the two vectors so that their initial points coincide. Then ${\bf v}$ and ${\bf w}$ form adjacent sides of a parallelogram, and the vector ${\bf v}+{\bf w}$ is the diagonal of the parallelogram, directed from the common initial point of ${\bf v}$ and ${\bf w}$ to the opposite vertex of the parallelogram, as shown below.	
\begin{center}
	\includegraphics[scale=0.5]{figures_mvc/parallelogram_rule}
\end{center}
This is called the \ti{parallelogram rule} for vector addition. Since the opposite sides of a parallelogram are congruent and parallel, we can equivalently view ${\bf v}+{\bf w}$ as the result of positioning the initial point of ${\bf w}$ at the terminal point of ${\bf v}$ and drawing the arrow connecting the initial point of ${\bf v}$ to the terminal point of ${\bf w}$.

\begin{center}
	\includegraphics[scale=0.5]{figures_mvc/tip_to_tail}
\end{center}
This is called the \ti{triangle rule} or \ti{``tip to tail" rule} for vector addition. These two points of view are related by \ti{parallel translation}. To go from the first point of view to the second, we translate the initial point of ${\bf w}$ along ${\bf v}$, keeping ${\bf w}$ parallel to its original direction at all times. Accordingly, ${\bf v}+{\bf w}$ is also called the \ti{translation of ${\bf w}$ by ${\bf v}$}.
\begin{center}
	\includegraphics[scale=0.5]{figures_mvc/translation_of_v_by_w}
\end{center}

\begin{example}
There is another physical interpretation of this addition rule which agrees with our intuition. Suppose a person walks 10 steps in a north-easterly direction, and then turns and walks another 5 steps to the east. The vector $\mathbf{v}$ then represents his \emph{displacement} from his initial position, with the length of $\mathbf{v}$ being his distance from where he started, and the direction of $\mathbf{v}$ pointing in the direction in which he moved. Similarly, the vector $\mathbf{w}$ represents his displacement from his position after he traveled along the vector $\mathbf{v}$. Their sum, added according to the tip to tail rule, is his \emph{total} displacement from his initial position.	
\end{example}

Let us now use this geometric picture to determine the properties of vector addition. Recall that the addition operation defined on the set of real numbers satisfies the following properties: \footnote{Any set $G$ on which there is a binary operation $*$ which satisfies the first three of these properties is said to form a \emph{group} under $*$. One also says that $(G,*)$ is a group, or just that $G$ is a group if $*$ is understood. If the fourth property (commutativity) also holds, $G$ is said to form a \emph{commutative} (or \emph{abelian}) under $*$.}
\begin{enumerate}[(i)]
	\item Associativity: $(x+y)+z=x+(y+z)$ for all $x,y,z \in \mathbb{R}$.
	\item Existence of an additive identity: $\mathbb{R}$ contains an element 0 such that $0+x=x$ for every $x \in \mathbb{R}$.
	\item Existence of additive inverses: For every $x \in \mathbb{R}$, there exists and element $y \in \mathbb{R}$ such that $x+y=0$.
	\item Commutativity: $x+y=y+x$ for all $x,y \in \mathbb{R}$.
\end{enumerate}
We now consider each of these in turn. We consider commutativity first, since it is the simplest to analyze.

\begin{prop}[Vector addition is commutative]
	Vector addition is commutative. That is, 
	\begin{align*}
		{\bf v}+{\bf w}={\bf w}+{\bf v}
	\end{align*}
	for any two vectors ${\bf v}$ and ${\bf w}$, since each of these is the diagonal of the parallelogram whose edges are formed by $\mathbf{v}$ and $\mathbf{w}$.
\end{prop}

\begin{pf}
	We see from the two diagrams below that the translation of ${\bf w}$ by ${\bf v}$ is the same vector as the translation of ${\bf v}$ by ${\bf w}$.
\begin{center}
	\includegraphics[scale=0.5]{figures_mvc/equivalence_of_vector_addition}
\end{center}

\end{pf}

\begin{prop}[Vector addition is associative]
	Vector addition is associative. That is, for any three vectors ${\bf u}, {\bf v},$ and ${\bf w}$, we have
	\begin{align*}
		{\bf u}+({\bf v}+{\bf w})=({\bf u}+{\bf v})+{\bf w}
	\end{align*}
We therefore denote both expressions by ${\bf u}+{\bf v}+{\bf w}$. 
\end{prop}

\begin{pf}
One can construct ${\bf u}+{\bf v}+{\bf w}$ by placing the vectors ``tip to tail" in succession and then drawing the vector from the initial point of ${\bf u}$ to the terminal point of ${\bf w}$. If the three vectors lie in the same plane, one can easily verify associativity from the diagram below.
\end{pf}

\begin{center}
	\includegraphics[scale=0.5]{figures_mvc/associativity_of_vector_addition}
\end{center}

If the three vectors do not all lie in the same plane, then when placed at the same initial point the vectors ${\bf u}, {\bf v},$ and ${\bf w}$ from adjacent edges of a \emph{parallelepiped}. \footnote{A parallelepiped is a polygon whose faces are parallelograms, with each pair of opposite sides parallel.} Translating these vectors and adding tip to tail, we see that ${\bf u}+{\bf v}+{\bf w}$ is the diagonal of this parallelepiped.

\begin{center}
	\includegraphics[scale=0.5]{figures_mvc/volume_of_parallelepiped_vectors}
\end{center}

\begin{cor}
The sum ${\bf v}_1+{\bf v}_2+\cdots + {\bf v}_k$ is independent of how the expression is bracketed.	
\end{cor}

\begin{pf}
We postpone the proof until we discuss vectors in coordinates in \fixme{Insert section reference here.}, as the geometry is very complicated and difficult to analyze.	
\end{pf}

Let us now check whether there is a vector which plays the role of an additive identity. That is, given any vector ${\bf b}$, is there a vector ${\bf 0}$ such that ${\bf b}+{\bf 0}={\bf b}$? Let us suppose there is such a vector ${\bf 0}$ and denote its magnitude by $r$. Let us now add ${\bf 0}$ to ${\bf b}$ by the tip-to-tail method. Since ${\bf 0}$ has magnitude $r$, ${\bf b}+{\bf 0}$ must lie on a circle of radius $r$ centered on the tip of ${\bf b}$. However, the condition ${\bf b}+{\bf 0}={\bf b}$ means that ${\bf b}+{\bf 0}$ must have the same magnitude and direction as ${\bf b}$, and there is no point on the circle for which this is true. This shows that there is no such vector ${\bf 0}$ with nonzero length. 
\begin{center}
	\includegraphics[scale=0.5]{figures_mvc/b_plus_zero_circle}
\end{center}

A vector of length zero, does not have a defined direction. Thus, there is a unique vector which acts as an additive identity with respect to vector addition.

\begin{defn}[The zero vector]
	The zero vector, which we denote by ${\bf 0}$, is the unique vector of magnitude zero.
\end{defn}

Finally, we want to investigate whether every vector has an additive inverse. That is, we want to determine if, when given any vector ${\bf a}$, we can find another vector ${\bf b}$ such that ${\bf a}+{\bf b}={\bf 0}$. Since the zero vector has zero length and since we add vectors from ``tip to tail", it follows that if ${\bf a}+{\bf b}={\bf 0}$, then the tail of ${\bf a}$ and the tip of ${\bf b}$ must coincide. Thus, any vector $\mathbf{a}$ has a \emph{unique} inverse, which we denote by $-\mathbf{a}$.   
\begin{center}
	\includegraphics[scale=0.5]{figures_mvc/a_plus_b_equals_zero}
\end{center} 
\begin{defn}[Inverse of a vector]
The additive inverse of a vector ${\bf a}$ is the unique vector $-\mathfrak{a}$, which has the same magnitude as $\mathbf{a}$ and opposite orientation.
\end{defn}
In analogy with the equation $x+(-x)=0$ for real numbers, we denote the additive inverse of ${\bf a}$ by $-{\bf a}$, so that ${\bf a}+(-{\bf a})={\bf 0}$ for any vector ${\bf a}$. We may again agree, as in the case of numerical addition, to abbreviate ${\bf a}+(-{\bf b})$ as ${\bf a}-{\bf b}$, which allows us to define vector subtraction:

\begin{defn}[Vector subtraction]
	The difference ${\bf a}-{\bf b}$ of two vectors ${\bf a}$ and ${\bf b}$ is the sum
	\begin{align*}
		{\bf a}-{\bf b}={\bf a}+(-{\bf b}).
	\end{align*}
\end{defn}
From this definition, we may view vector subtraction geometrically as follows: To form ${\bf a}-{\bf b}$,
\begin{enumerate}[(1)]
	\item Obtain $(-{\bf b})$ from ${\bf b}$ by reversing the direction of ${\bf b}$.
	\item Add ${\bf a}$ and $(-{\bf b})$ in the usual way, by placing the tail of $(-{\bf b})$ at the tip of ${\bf a}$. \footnote{Like numerical subtraction, vector subtraction is \ti{not} commutative, so the order matters here.}
\end{enumerate}

\begin{center}
	\includegraphics[scale=0.5]{figures_mvc/a_minus_b}
\end{center}

\begin{exercise}
Consider adding vectors ${\bf a}$ and ${\bf b}$ tail-to-tail, as shown below
	\begin{center}
		\includegraphics[scale=0.5]{figures_mvc/b_plus_x_equals_a}
	\end{center}
	Show that ${\bf x}={\bf a}-{\bf b}$. This gives another view of vector subtraction which is very useful computationally.
\end{exercise}

{\color{red} \flushleft {\bf Solution:}
We see from the tip-to-tail rule, that ${\bf b}+{\bf x}={\bf a}$. Adding $(-{\bf b})$ to both sides of this equation and using associativity gives ${\bf x}={\bf a}-{\bf b}$.}

The previous exercise shows that, if ${\bf a}$ and ${\bf b}$ are any two vectors and we wish to find ${\bf a}-{\bf b}$, we place the two vectors tail-to-tail and ${\bf a}-{\bf b}$ is then the vector which extends from the tip of ${\bf b}$ to the tip of ${\bf a}$.

We have just shown that our definition of vector addition satisfies the same properties as numerical addition. That is, the additive structures are exactly the same for vectors as for numbers. \footnote{To state this more formally, they both form an abelian group.} Therefore, any results that hold for numbers also hold for vectors, for exactly the same reasons. For example,

\begin{thm}[Cancellation law]
	If ${\bf a}, {\bf b},$ and ${\bf c}$ are vectors such that ${\bf a}+{\bf b}={\bf a}+{\bf c}$, then ${\bf b}={\bf c}$. 
\end{thm}

\begin{pf}
The proof for real numbers is as follows: let $x,y,z$ be real numbers such that 
\begin{align*}
	x+y=x+z
\end{align*}
Adding $-x$ to both sides of this equation, we have
\begin{align*}
	-x+(x+y)&=-x+(x+z) \\
	(-x+x)+y&=(-x+x)+z \text{ (since $+$ is associative)} \\
	0+y&=0+z \text{ (since $-x$ is the inverse of $x$)} \\
	y&=z \text{ (since $0$ is the additive identity)}
\end{align*}
Since addition of vectors obeys exactly the same properties as addition of real numbers, this same proof holds for vectors simply by drawing arrows over $x,y$ and $z$!
\end{pf}

\subsubsection{Scalar multiplication}
In physics, we observe that an unbalanced force on a body causes an acceleration in the direction of the force. It is also observed that the magnitude of acceleration of different bodies, when subjected to the same force, varies according to their mass. These observations are formalized in Newton's second law of motion
\begin{equation}
	{\bf F}=m{\bf a}.
\end{equation}
On the right side of this equation, we see a new operation: the product of a scalar and a vector. 

\begin{defn}[Scalar multiplication]
Let $\mathbf{v}$ be a vector and $k$ a scalar. The \emph{scalar multiple} of ${\bf v}$ by $k$ is a vector $\mathbf{w}$, defined as follows:
\begin{itemize}
	\item The length of ${\bf w}$ is $|k|$ times the length of ${\bf v}$. If $|k|=0$, then $\mathbf{w}$ is the zero vector.
	\item $\mathbf{w}$ is parallel to $\mathbf{v}$.
	\item If $k>0$, then $\mathbf{w}$ has the same orientation as $\mathbf{v}$. If $k<0$, then $\mathbf{w}$ and $\mathbf{v}$ have opposite orientation.
\end{itemize}
If $\mathbf{w}$ is the scalar multiple of $\mathbf{v}$ by $k$, we write $\mathbf{w}=k\mathbf{v}$.
\end{defn}

Note that scalar multiplication is not a binary operation on $V$, since it does not take two vectors to a vector, but rather a scalar and a vector to a vector. More formally, scalar multiplication is a map $\R \times V \to V$.

\begin{example}
The vector $2{\bf v}$ has the same direction as ${\bf v}$ but twice its length, while $-2{\bf v}$ is oppositely directed to ${\bf v}$ and twice its length. \fixme{Insert graphic.}
\end{example}

\begin{prop}[Properties of scalar multiplication]
For any scalars $c,d$ and vectors ${\bf v}, {\bf w}$, 
	\begin{enumerate}[(i)]
		\item $(c+d){\bf v}=c{\bf v}+d{\bf v}$,
		\item $c({\bf v}+{\bf w})=c{\bf v}+c{\bf w}$,
		\item $(dc){\bf v}=c(d{\bf v})$,
		\item 1{\bf v}={\bf v}.
	\end{enumerate}
\end{prop}

\begin{pf}
	\fixme{Add proof. Does 1 depend on the triangle inequality?}
\end{pf}

\begin{center}
	\includegraphics[scale=0.5]{figures_mvc/simtri}
\end{center}

\begin{cor}
The following properties hold:
	\begin{enumerate}[(a)]
		\item 0{\bf v}={\bf 0},
		\item $(-1){\bf v}=-{\bf v}$.
	\end{enumerate}
\end{cor}

\subsection{Vectors in coordinate systems}
\subsubsection{Vectors in one dimension}
Consider a particle \footnote{By \emph{particle}, I mean an idealized object which is so small that all of its matter is concentrated at a single point in space.} constrained to move along a line. That is, it can only move to the left or right, and cannot move in any other direction. The position of the particle at any instant is uniquely specified by a single real number, $x$, which is the coordinate of the particle relative to some chosen origin. Since the coordinate of the particle consists of a single number, the motion along the line is said to be \emph{one-dimensional} or in \emph{one dimension}. \footnote{We will give a more precise definition of the notion of dimension in section \fixme{Insert section number.}} 

If $x \neq 0$ (that is, if the particle is not at the origin), we can rewrite $x$ as 
\begin{equation}\label{eq:1_d_vector}
	x=\frac{x}{|x|}|x|
\end{equation}
where 
\begin{align*}
	|x|=\begin{cases}
		x, &\text{ if } x \geq 0, \\
		-x, &\text{ if } x<0,
	\end{cases}
\end{align*}
is the absolute value of $x$ and 
\begin{align*}
	\frac{x}{|x|}=\begin{cases}
		+1, &\text{ if } x>0, \\
		-1, &\text{ if } x<0 \\
	\end{cases}
\end{align*}
is the sign of $x$, which we will denote by $\sgn(x)\equiv \frac{x}{|x|}$. Geometrically, $|x|$ is the \emph{distance} of the particle from the origin, while $\sgn(x)$ is the \emph{orientation} of the particle; that is, $\sgn(x)=+1$ if the particle is to the right of the origin, and $\sgn(x)=-1$ if the particle is to the left of the origin. We see that the \emph{displacement vector} of a particle moving in one dimension, which we denote by $\mathbf{x}$, is given the ordered pair 
\begin{align*}
	\mathbf{x}=(|x|,\sgn(x))
\end{align*}
if $x \neq 0$. Since $\sgn(x)$ is not defined if $x=0$, if the particle is at the origin the displacement vector is given by the single number $\mathbf{x}=0$. In either case, equation \eqref{eq:1_d_vector} shows that the displacement vector $\mathbf{x}$ is equivalent to the coordinate $x$ of the particle.

Geometrically, we can visualize a nonzero vector $\mathbf{x}$ by an arrow of length $|x|$ which originates at the origin and terminates at the point $x$. \fixme{Insert graphic.}

The velocity, $\mathbf{v}$, of the particle is specified by its speed and direction of motion. E.g., 25 m/s in the positive direction. \footnote{The standard choice is to take the \emph{positive direction} to be the direction of \emph{increasing} $x$; the \emph{negative direction} is then the direction of \emph{decreasing} $x$.} As in equation \eqref{eq:1_d_vector}, if the speed is nonzero we can encode this information into a single real number $v=\frac{v}{|v|}|v|$, as 
\begin{align*}
	\mathbf{v}=(|v|,\sgn(v))
\end{align*}
where $|v|$ is the speed of the particle and $\sgn(v)$ is taken to be $+1$ if the particle is moving in the positive direction and $-1$ if the particle is moving in the negative direction. We can again visualize $\mathbf{v}$ as an arrow originating at the origin of the number line and terminating at the number $v$, except in this case the points on the number line represent velocities, and not points along the physical line along which the particle is moving.

Similarly, the acceleration vector $\mathbf{a}$  specified in terms of the real number $a=\frac{a}{|a|}|a|$ (if $a \neq 0$) by
\begin{align*}
	\mathbf{a}=(|a|,\sgn(a))
\end{align*}
where $|a|$ is the magnitude of the acceleration and $\sgn(a)$ is taken to be $+1$ if the particle is speeding up in the positive direction and $-1$ if it is slowing down in the positive direction (which is the same as speeding up in the negative direction). As before, we can visualize the vector $\mathbf{a}$ as an arrow of length $|a|$ originating at the origin and terminating at the point $a$.

Indeed, \emph{any} nonzero vector $\mathbf{t}$ in one dimension whose magnitude is nonzero can be specified in terms of a real number 
\begin{equation}\label{eq:general_1_d_vector_1}
	t=\frac{t}{|t|}|t|
\end{equation}
by
\begin{equation}\label{eq:general_1_d_vector_2}
	\mathbf{t}=(|t|,\sgn(t))
\end{equation} 
where $|t|$ is the magnitude of $\mathbf{t}$ and, since $\mathbf{t}$ has only one of two  possible orientations in one dimension, we assign $\sgn(t)=+1$ to one of these orientations and $\sgn(t)=-1$ to the other. As in the previous examples, the vector $\mathbf{t}$ is visualized as an arrow originating from the origin and terminating at the real number $t$. A vector with zero magnitude has an undefined direction, and is simply represented by the real number 0. 

Together with equations \eqref{eq:general_1_d_vector_1} and \eqref{eq:general_1_d_vector_2}, this shows that vectors in one-dimension are in 1-1 correspondence with real numbers. In section \ref{sec:scalars}, we saw that scalars are also in 1-1 correspondence with real numbers, so \emph{there is a 1-1 correspondence between one-dimensional vectors and scalars}. Thus there is no essential difference between between a vector in one dimension and a scalar, since each a one-dimensional vector is associated to a unique scalar, and vice versa. The difference between vectors and scalars only arises in higher dimensions, as we will see next.

\subsubsection{Vectors in  two dimensions}
Let us now consider motion restricted to a \emph{plane}, rather than a line. Since each point in a plane specified by \emph{two} coordinates $(x,y)$ with respect to some chosen origin, we say this motion is \emph{two-dimensional}. 

Let's again consider the displacement vector, $\mathbf{x}$, of the particle. If the particle is not at the origin, this vector, as before, can be visualized as an arrow originating at the origin and terminating at the point $(x,y)$. The length of the arrow is the magnitude of $\mathbf{x}$, which is the distance of the particle from the origin. Denoting the magnitude of $\mathbf{x}$ by $||\mathbf{x}||$, \footnote{The double vertical bars are to differentiate this from the absolute value in one dimension.}
this is given by the distance formula
\begin{equation}\label{eq:pythagorean}
	||\mathbf{x}||=\sqrt{x^2+y^2}.
\end{equation}
We can specify the direction of the arrow by giving the angle it makes with respect to the positive $x$-axis, which is given by
\begin{align*}
	\theta=\tan^{-1}\left(\frac{y}{x}\right).
\end{align*}
Note that this is just the usual conversion from rectangular to polar coordinates. By \eqref{eq:pythagorean}, $||\mathbf{x}||=0$ if and only if $(x,y)=(0,0)$, so we see that the possible displacement vectors of a particle in two dimensions are in 1-1 correspondence with points in $\mathbb{R}^2$.

Note that if the coordinate $y=0$ so that the point $(x,0)$ lies on the $x$-axis (so that $\mathbf{x}$ is effectively a one-dimensional vector), we have

\begin{align*}
	||\mathbf{x}||&=\sqrt{x^2+0^2} \\
	&=\sqrt{x^2} \\
	&=|x|,
\end{align*}
in agreement with the previous section.

\begin{exercise}
Verify that $\sqrt{x^2}$ is indeed equal to $|x|$.	
\end{exercise}

Let us now consider \emph{any} vector $\mathbf{t}$ in two-dimensions, which is specified by a nonnegative real number $||\mathbf{t}||$ and an angle $\theta \in [0,2\pi)$. We can associate to $\mathbf{t}$ a unique ordered pair of real numbers $(x,y)$ by taking
\begin{align*}
	\mathbf{t}=(||\mathbf{t}||,\theta)=\begin{cases}
		(\sqrt{x^2+y^2}, \tan^{-1}\left(\frac{y}{x}\right)), &\text{ if } ||\mathbf{t}|| \neq 0, \\
		(0,0), &\text{ if } ||\mathbf{t}||=0.
	\end{cases}
\end{align*}
Thus, vectors in two dimensions are in 1-1 correspondence with points in $\mathbb{R}^2$. Given a vector $\mathbf{t}$, we will denote the corresponding ordered pair of real numbers by $\mathbf{t}=(t_1,t_2)$. These numbers are called the \emph{components} of the vector $\mathbb{t}$ in our chosen coordinate system.

As in one-dimension, for a given vector $\mathbf{t}$ the corresponding point in $\mathbb{R}^2$ does not necessarily represent a point in the physical space in which the particle is moving (which is also represented by $\mathbb{R}^2$). For example, if $\mathbf{v}=(v_1,v_2)$ is the velocity of the particle. Then $v_1$ is the component of velocity along the $x$-direction while $v_2$ is the component of velocity along the $y$-direction. These do not correspond to the location of the particle, and if we are only told its velocity, the particle may indeed be located anywhere in space for all we know.

\subsubsection{Vectors in three dimensions}
Finally, consider a particle moving in three-dimensional space. The position of the particle is now specified by an ordered triple of real numbers $(x,y,z)$, which specifies its coordinates with respect to three mutually perpendicular coordinate axes. As before, we visualize the displacement vector of the particle as an arrow originating from the origin and terminating at the point $(x,y,z)$.

\begin{figure}[h]
	 \begin{center}
	\includegraphics[scale=0.6]{figures_mvc/components}
\end{center}
\caption{Displacement vector in three dimensions.}
\end{figure}

The set of all ordered triples of real numbers is denoted $\mathbb{R}^3$.  Indeed, any vector $\mathbf{t}$ in three dimensions is uniquely specified by a point in $\mathbb{R}^3$, $\mathbf{t}=(t_1,t_2,t_3)$ called the \emph{components} of $\mathbf{t}$. The correspondence for nonzero $||\mathbf{t}||$ is given as follows. A direction in three-dimensions is specified by two angles, which we can take to be the angle $\mathbf{t}$ makes with respect to the positve $z$-axis, which we denote by $\varphi$, and the angle that the \emph{projection} \footnote{We will define precisely what this means later, but for now think of this as the shadow of the vector on the $xy$-plane you would see if the sun were directly overhead.} of $\mathbf{t}$ makes with respect to the positive $x$-axis in the $xy$-plane, which we denote by $\theta$ since it is the usual polar angle. Then a little trigonometry shows that the point $(x,y,z)$ has coordinates
\begin{align*}
	x &=||\mathbf{t}||\sin \varphi \cos \theta, \\
	y &=||\mathbf{t}||\sin \varphi \sin\theta, \\
	z &=||\mathbf{t}||\cos \varphi.
\end{align*}
So, given any vector $\mathbf{t}=(||\mathbf{t}||,\varphi,\theta)$ in three dimensions with $||\mathbf{t}||\neq 0$, the correspondence is given by
\begin{align*}
	||\mathbf{t}||&=\sqrt{x^2+y^2+z^2}, \\
	\varphi &= \cos^{-1}\left(\frac{z}{||\mathbf{t}||}\right), \\
	\theta &=\frac{x}{||\mathbf{t}||\sin \varphi}.
\end{align*}
Thus, vectors in three dimensions are in 1-1 correspondence with points in $\mathbb{R}^3$.



\subsubsection{Norm of a vector}
A \ti{norm} is a function which takes a vector as input and returns its magnitude. Here, we review the standard \ti{Euclidean norm}, which is given in terms of the Cartesian coordinates of a vector. 
\begin{defn}[Norm of a vector]\label{def:length_of_a_vector}
	The \ti{norm} of a vector ${\bf v}$, denoted $||{\bf v}||$, is given in terms of its Cartesian coordinates by
	\begin{align*}
		||{\bf v}||=\sqrt{v_1^2+v_2^2+v_3^2}.
	\end{align*}
\end{defn}

\begin{pf}
This formula follows directly from the Pythagorean theorem. In the diagram below,
\begin{center}
	\includegraphics[scale=0.5]{figures_mvc/euclidean_norm}
\end{center}	
we see from the right triangle in the $x_1x_2$-plane that that $s^2=x_1^2+x_2^2$. Then, from triangle $XYO$, we have $r^2=s^2+x_3^2=x_1^2+x_2^2+x_3^2$.
\end{pf}

\begin{prop}[Norm properties]
	For any scalar $a$ and any vectors ${\bf v}$ and ${\bf w}$,
	\begin{enumerate}[(1)]
		\item $||{\bf u}+{\bf v}||\leq ||{\bf u}|| + ||{\bf v}||$ (triangle inequality)
		\item $||a{\bf v}||=|a| \ ||{\bf v}||$ (homogeneity)
		\item If $||{\bf v}||=0$, then ${\bf v}={\bf 0}$ (positive-definite)
	\end{enumerate}
\end{prop}

\begin{pf}
Verify these for the Euclidean norm. More generally, we take these to be the defining properties of a norm.
\end{pf}

\begin{defn}[Unit vector]
	A \ti{unit vector} is a vector of unit length. That is, a vector ${\bf v}$ such that $||{\bf v}||=1$.
\end{defn}

\begin{prop}[Normalizing a vector]
	If ${\bf v}\neq 0$, then ${\bf v}/||{\bf v}||$ is a unit vector in the direction of ${\bf v}$. The unit vector ${\bf v}/||{\bf v}||$ is often denoted ${\bf \hat{v}}$ and the process of forming ${\bf \hat{v}}$ from ${\bf v}$ is called \ti{normalizing} ${\bf v}$.
\end{prop}

\begin{pf}
Since ${\bf v}\neq 0$, $||{\bf v}|| \neq 0$, so we can multiply ${\bf v}$ by $1/||{\bf v}||$. Computing the length of $||\frac{{\bf v}}{||{\bf v}||}||$, we see that 
\begin{align*}
	||\frac{{\bf v}}{||{\bf v}||}||&=\frac{||{\bf v}||}{||{\bf v}||}=1,
\end{align*} 
hence ${\bf v}/||{\bf v}||$ is a unit vector. Since ${\bf v}/||{\bf v}||$ is a scalar multiple of ${\bf v}$ (by $k=1/||{\bf v}||$) it is parallel to ${\bf v}$.
\end{pf}

\begin{example}
	We can express any nonzero vector ${\bf v}$ as a product of its magnitude and direction by means of the formula
	\begin{align*}
		{\bf v}=||{\bf v}||\frac{{\bf v}}{||{\bf v}||}.
	\end{align*}
	For example, taking ${\bf v}=(1,-2,3)$,
	\begin{align*}
		||{\bf v}||=\sqrt{1^2+(-2)^2+3^2}=\sqrt{14}
	\end{align*}
	and thus
	\begin{align*}
		{\bf v}=\sqrt{14}(\frac{1}{\sqrt{14}},\frac{-2}{\sqrt{14}},\frac{3}{\sqrt{14}}).
	\end{align*}
\end{example}


\begin{defn}[Standard unit vectors]
	 The \ti{standard unit vectors} for $\mathbb{E}^3$ are the vectors
	\begin{align*}
		{\bf e}_1&=(1,0,0), \\
		{\bf e}_2&=(0,1,0), \\
		{\bf e}_3&=(0,0,1). \\
	\end{align*}
These vectors are unit vectors pointing along the $x$-, $y$-, and $z$-axes, respectively. \footnote{The standard unit vectors $({\bf e}_1,{\bf e}_2,{\bf e}_3)$ are also commonly denoted as $({\bf i}, {\bf j}, {\bf k})$ or $({\bf \hat{x}}, {\bf \hat{y}}, {\bf \hat{z}})$.}	
\end{defn}
Using the standard unit vectors, we may write the vector ${\bf v}=(v_1,v_2,v_3)$ as $${\bf v}=v_1{\bf e}_1+v_2{\bf e}_2+v_3{\bf e}_3.$$

\begin{prop}[Vector operations in coordinates]
	Let ${\bf v}=(v_1,v_2,v_3)$ and ${\bf w}=(w_1,w_2,w_3)$ be vectors and $k$ a scalar. Then
	\begin{align*}
		{\bf v}+{\bf w}&=(v_1+w_1,v_2+w_2,v_3+w_3), \\
		{\bf v}-{\bf w}&=(v_1-w_1,v_2-w_2,v_3-w_3), \\
		k{\bf v}&=(kv_1,kv_2,kv_3).
	\end{align*}
\end{prop}

\begin{pf}
These formulas can be seen by examining the following diagrams.
\begin{center}
	\includegraphics[scale=0.5]{figures_mvc/vector_operations}
\end{center}	
\end{pf}

\begin{prop}[Vector from one point to another]\label{prop:vector_from_one_point_to_another}
	The vector from $P_1(x_1,y_1,z_1)$ to $P_2(x_2,y_2,z_2)$ is given by
	\begin{align*}
		\overrightarrow{P_1P_2}=(x_2-x_1,y_2-y_1,z_2-z_1).
	\end{align*}
\end{prop}

\begin{pf}
From the figure below, 
\begin{center}
	\includegraphics[scale=0.5]{figures_mvc/P1P2}
\end{center}

we see that $\overrightarrow{OP_1}+\overrightarrow{P_1P_2}=\overrightarrow{OP_2}$, and therefore
\begin{align*}
	\overrightarrow{P_1P_2}&=\overrightarrow{OP_2}-\overrightarrow{OP_1} \\
	&=(x_2,y_2,z_2)-(x_1,y_1,z_1) \\
	&=(x_2-x_1,y_2-y_1,z_2-z_1).
\end{align*}	
\end{pf}

\begin{cor}[Distance between two points]
	The distance between two points $P_1(x_1,y_1,z_2)$ and $P_2(x_2,y_2,z_2)$ is given by
	\begin{align*}
		||\overrightarrow{P_1P_2}||=\sqrt{(x_2-x_1)^2+(y_2-y_1)^2+(z_2-z_1)^2}.
	\end{align*}
\end{cor}

\begin{pf}
	This formula follows immediately from Prop. \ref{prop:vector_from_one_point_to_another} and Def. \ref{def:length_of_a_vector}.
\end{pf}

\begin{exercise}
\begin{enumerate}[(a)]
	\item Find the components of the vector $\overrightarrow{P_1P_2}$ with initial point $P_1(2,-1,4)$ and terminal point $P_2(7,5,-8)$.
	\item Find the distance between $P_1(2,-1,4)$ and $P_2(7,5,-8)$.
\end{enumerate}	
\end{exercise}

{\color{red} \flushleft {\bf Solution:} 
\begin{enumerate}[(a)]
	\item The components of $\overrightarrow{P_1P_2}$ are given by
	\begin{align*}
		\overrightarrow{P_1P_2}=(7-2,5-(-1),-8-4)=(5,6,-12)
	\end{align*}
	\item The distance between $P_1$ and $P_2$ is 
	\begin{align*}
		||\overrightarrow{P_1P_2}||=\sqrt{5^2+6^2+(-12)^2}=\sqrt{205}.
	\end{align*}
\end{enumerate}}

\begin{exercise}
Find a unit vector in the direction of the vector from $P_1(1,0,1)$ to $P_2(3,2,0)$.
\end{exercise}

{\color{red} \flushleft {\bf Solution:}
We find $\overrightarrow{P_1P_2}$ and normalize:
\begin{align*}
	\overrightarrow{P_1P_2}&=(3-1,2-0,0-1)=(2,2,-1), \\
	||\overrightarrow{P_1P_2}||&=\sqrt{2^2+2^2+(-1)^2}=3, \\
\end{align*}
and therefore the desired unit vector is given by
\begin{align*}
	\frac{\overrightarrow{P_1P_2}}{||\overrightarrow{P_1P_2}||}=(\frac{2}{3},\frac{2}{3},-\frac{1}{3}).
\end{align*}}

\begin{exercise}
Find a vector 6 units long in the direction of ${\bf v}=(2,2,-1)$.	
\end{exercise}

{\color{red} \flushleft {\bf Solution:}
The vector we want is 
\begin{align*}
6\frac{{\bf v}}{||{\bf v}||}=6\frac{(2,2,-1)}{\sqrt{2^2+2^2+(-1^2)}}=6\frac{(2,2,-1)}{3}=(4,4,-2).	
\end{align*}
}

\begin{exercise}
	A sphere of radius $r$ centered at $(x_0,y_0,z_0)$ is the set of all points in $\mathbb{R}^3$ equidistant from $(x_0,y_0,z_0)$. Find the equation of the sphere.
\end{exercise}

{\color{red} \flushleft {\bf Solution:}
Let $(x,y,z)$ be an arbitrary point in $\mathbb{R}^3$. The vector from $P_1(x_0,y_0,z_0)$ to $P_2(x,y,z)$ is then
\begin{align*}
	\overrightarrow{P_1P_2}=(x-x_0,y-y_0,z-z_0).
\end{align*}
The length of $\overrightarrow{P_1P_2}$ is then
\begin{align*}
	||\overrightarrow{P_1P_2}||=\sqrt{(x-x_0)^2+(y-y_0)^2+(z-z_0)^2}.
\end{align*}
By definition, the sphere is the set of all $(x,y,z)$ such that $||\overrightarrow{P_1P_2}||=r$. Squaring both sides to get rid of the awkward square root gives the standard equation of a sphere:
\begin{align*}
	(x-x_0)^2+(y-y_0)^2+(z-z_0)^2=r^2.
\end{align*}}

\begin{prop}[Midpoint formula]
	The midpoint $M$ of the line segment joining points $P_1(x_1,y_1,z_1)$ and $P_2(x_2,y_2,z_2)$ is the point
	\begin{align*}
		\left(\frac{x_1+x_2}{2},\frac{y_1+y_2}{2},\frac{z_1+z_2}{2}\right).
	\end{align*}
\end{prop}

\begin{pf}
From the figure below, we see that
\begin{align*}
	\overrightarrow{OM}&=\overrightarrow{OP_1}+\frac{1}{2}\overrightarrow{P_1P_2} \\
	&=\overrightarrow{OP_1}+\frac{1}{2}(\overrightarrow{OP_2}-\overrightarrow{OP_1}) \\
	&=\frac{1}{2}(\overrightarrow{OP_2}+\overrightarrow{OP_1}) \\
	&=\left(\frac{x_1+x_2}{2},\frac{y_1+y_2}{2},\frac{z_1+z_2}{2}\right).
\end{align*}
\begin{center}
	\includegraphics[scale=0.5]{figures_mvc/midpoint_formula}
\end{center}	
\end{pf}

\begin{exercise}
Find the midpoint of the segment joining $P_1=(3,-2,0)$ and $P_2(7,4,4)$.	
\end{exercise}

{\color{red} \flushleft {\bf Solution:}
The midpoint is the point
\begin{align*}
	\left(\frac{3+7}{2},\frac{-2+4}{2},\frac{0+4}{2}\right)=(5,1,2).
\end{align*}}

\subsubsection{The dot product}
Two nonzero vectors ${\bf u}$ and ${\bf v}$ positioned so that their initial points coincide determine an angle $\theta \in [0,\pi]$, which is the angle between the two vectors.
\begin{center}
	\includegraphics[scale=0.5]{figures_mvc/angle_uv}
\end{center}
Note that the information about $\theta$ is encoded in ${\bf u}-{\bf v}$, since if we fix the magnitudes of ${\bf u}$ and ${\bf v}$ and open the angle, then ${\bf u}-{\bf v}$ will also change. The fundamental relation satisfied by ${\bf u}, {\bf v}$ and $\theta$ is the law of cosines, which says that if ${\bf w}={\bf u}-{\bf v}$, then
\begin{equation}\label{eq:law_of_cosines}
	||{\bf w}||^2=||{\bf u}||^2+||{\bf v}||^2-2||{\bf u}|| \ ||{\bf v}||\cos \theta. 
\end{equation}
Let us make the following definition.
\begin{defn}[Dot product]
	The \ti{dot product} of two vectors ${\bf u}$ and ${\bf v}$ is defined by
	\begin{equation}\label{eq:dot_product_def}
		{\bf u}\cdot {\bf v}=||{\bf u}|| \ ||{\bf v}||\cos\theta.
	\end{equation} 
\end{defn}
The angle between two vectors ${\bf u}$ and ${\bf v}$ is then given by
\begin{equation}\label{eq:angle_uv}
	\theta=\cos^{-1}\left(\frac{{\bf u}\cdot {\bf v}}{||{\bf u}|| \ ||{\bf v}||}\right),
\end{equation}	
and we see that
\begin{itemize}
	\item $\theta$ is acute if ${\bf u} \cdot {\bf v} > 0$.
	\item $\theta$ is obtuse if ${\bf u} \cdot {\bf v} < 0$.
	\item $\theta$ is right if ${\bf u} \cdot {\bf v} = 0$.
\end{itemize}

Since the dot product takes as input two vectors and returns a scalar, it is also called the \ti{scalar product}. Note that from Eq. \eqref{eq:law_of_cosines}, we can write ${\bf u}\cdot {\bf v}$ in terms of magnitudes only, as 
\begin{equation}\label{eq:dot_product_magnitudes}
	{\bf u}\cdot {\bf v}=\frac{||{\bf u}||^2+||{\bf v}||^2-||{\bf u}-{\bf v}||^2}{2}.
\end{equation}
Note that this equation holds in \ti{all} coordinate systems. However, it takes a particularly simple form in Cartesian coordinates. Writing out the norm of each vector in terms of its components gives

\begin{align*}
	{\bf u}\cdot {\bf v}&=\frac{u_1^2+u_2^2+u_3^2+v_1^2+v_2^2+v_3^2-(u_1-v_1)^2-(u_2-v_2)^2-(u_3-v_3)^2}{2}
\end{align*}
Expanding the binomials and cancelling like terms, we find
\begin{equation}
	{\bf u}\cdot {\bf v}=u_1v_1+u_2v_2+u_3v_2.
\end{equation}

\begin{exercise}
Compute the angle between the vectors ${\bf u}=(0,0,3)$ and ${\bf v}=(\sqrt{2},0,\sqrt{2})$.	
\end{exercise}

{\color{red} \flushleft {\bf Solution:} The angle between these vectors is
\begin{align*}
	\theta=\cos^{-1}\left(\frac{0(\sqrt{2}+0(0)+3(\sqrt{2}))}{3(2)}\right)=\cos^{-1}\left(\frac{1}{\sqrt{2}}\right)=\frac{\pi}{4}.
\end{align*}}

\begin{exercise}
	Find the angle between a diagonal of a cube and one of its edges.
\end{exercise}

{\color{red} \flushleft {\bf Solution:}
Let $s$ be the length of an edge and place the cube in the first octant so that one vertex is at the origin and two edges are along the $x$- and $y$-axes.
\begin{center}
	\includegraphics[scale=0.5]{figures_mvc/cube_coordinates}
\end{center}
If we let ${\bf u}_1=(s,0,0), {\bf u}_1=(0,s,0)$, and ${\bf u}_3=(0,0,s)$, then the vector
\begin{align*}
	{\bf d}=(s,s,s)={\bf u}_1+{\bf u}_2+{\bf u}_3
\end{align*}
is a diagonal of the cube. The angle between ${\bf d}$ and ${\bf u}_1$ is 
\begin{align*}
	\theta &=\cos^{-1}\left(\frac{{\bf u}_1 \cdot {\bf d}}{||{\bf u}_1 || \ || {\bf d}||}\right) \\
	&=\cos^{-1}\left(\frac{s^2}{(s)(\sqrt{3s^2})}\right)  \\
	&=\cos^{-1}\left(\frac{1}{\sqrt{3}}\right) \\
	&\approx 54.74^\circ.
\end{align*}
}

\begin{prop}[Properties of the dot product]
	Let ${\bf a}, {\bf b}, {\bf c}, {\bf d}$ be vectors and $k$ any scalar. Then 
	\begin{enumerate}[(1)]
		\item ${\bf a}\cdot {\bf b}={\bf b} \cdot {\bf a}$
		\item $(k {\bf a})\cdot {\bf b}={\bf a}\cdot (k{\bf b})=k({\bf a}\cdot {\bf b})$
		\item ${\bf a}\cdot ({\bf b}+{\bf c})={\bf a}\cdot {\bf b}+{\bf a}\cdot {\bf c}$
		\item $({\bf a}+{\bf b})\cdot {\bf a}={\bf a}\cdot {\bf c}+{\bf b}\cdot {\bf c}$
		\item $({\bf a}+{\bf b})\cdot ({\bf c}+{\bf d})={\bf a}\cdot {\bf c}+{\bf a}\cdot {\bf d}+{\bf b}\cdot {\bf c}+{\bf b}\cdot {\bf d}$
		\item ${\bf a} \cdot {\bf a}=||{\bf a}||^2$
	\end{enumerate}
\end{prop}

\begin{pf}
Each of these can be proved by writing out the vectors in components. For example, to prove (1)
\begin{align*}
	{\bf a}\cdot {\bf b}&=a_1b_1+a_2b_2+a_3b_3 \\
	&=b_1a_1+b_2a_2+b_3a_3 \\
	&={\bf b}\cdot {\bf a}
\end{align*}
Properties (2)-(6) are proved similarly. Note that (5) follows from (3) and (4), but is included for emphasis since it is used frequently. 	
\end{pf}

\fixme{Add exercises using these properties.}

Note, however, the differences between the dot product and ordinary multiplication. For instance, one might ask, ``Is the dot product associative?". This question doesn't even make any sense for the dot product, as expressions such as ${\bf a}\cdot ({\bf b}\cdot {\bf c})$ are not defined, since ${\bf a}$ is a vector and $({\bf b}\cdot {\bf c})$ is a scalar, and one can only form the dot product between two vectors. 

\subsubsection{Orthogonal vectors}
While writing vectors in component form has the advantage of facilitating many computations, this form seems to obscure geometric relations between vectors.
For instance, the two vectors
\begin{align*}
	{\bf u}&=(3,-2,1) \\
	{\bf v}&=(0,2,4).
\end{align*}
are orthogonal (perpendicular), but this does not seem obvious in the given form. We know from right triangle trigonometry that right angles are very special, so it would be nice to have a way to check if two vectors written in component form are perpendicular. Fortunately, the dot product gives us an easy way to determine this. 
\begin{prop}[Orthogonal vectors]\label{prop:orthogonal_vectors}
	Two nonzero vectors ${\bf a}$ and ${\bf b}$ are orthogonal if and only if ${\bf a}\cdot {\bf b}=0$.
\end{prop}

\begin{pf}
	The dot product of ${\bf a}$ and ${\bf b}$ is given by
\begin{align*}
	{\bf a}\cdot {\bf b}=||{\bf a}|| \ ||{\bf b}||\cos \theta.
\end{align*}	
If ${\bf a}$ and ${\bf b}$ are non-zero, then $||{\bf a}||$ and $||{\bf b}||$ are greater than zero, so ${\bf a}\cdot {\bf b}=0$ if and only if $\theta=\frac{\pi}{2}$, that is, if and only if ${\bf a}$ and ${\bf b}$ are orthogonal. 
\end{pf}

What if ${\bf b}={\bf 0}$? In this case, $\theta$ is not well-defined, since we have defined ${\bf 0}$ to be any vector with zero magnitude, independently of direction. For Proposition \ref{prop:orthogonal_vectors} to hold even if ${\bf a}$ or ${\bf b}$ are ${\bf 0}$, we simply define ${\bf a}\cdot {\bf b}=0$ if one of the vectors is the zero vector. With this definition, the zero vector is orthogonal to every vector, including itself.

\begin{example}
The two vectors in the example above	
\begin{align*}
	{\bf u}&=(3,-2,1) \\
	{\bf v}&=(0,2,4).
\end{align*}
are orthogonal since ${\bf u}\cdot {\bf v}=3(0)-2(2)+1(4)=-4+4=0$.
\end{example}

\begin{example}
	The standard unit vectors for $\mathbb{R}^3$ are mutually orthogonal. One can easily verify that 
	\begin{align*}
		{\bf e}_i \cdot {\bf e_j}=\delta_{ij}
	\end{align*}
where 
\begin{align*}
	\delta_{ij} \equiv \begin{cases}
		1 \text{ if } i=j, \\
		0 \text{ if } i \neq j
	\end{cases}
\end{align*}
is called the \ti{Kronecker delta symbol}.
\end{example}

These examples illustrate another difference between the dot product of two vectors and the ordinary product of two numbers. For two real numbers, if $ab=0$, then either $a=0$ or $b=0$. These examples clearly show that if ${\bf a}\cdot {\bf b}=0$, then it need not be true that either ${\bf a}=0$ or ${\bf b}=0$.
\subsubsection{Projection of a vector}
Let ${\bf a}$ be a nonzero vector, ${\bf \hat{a}}=\frac{{\bf a}}{||{\bf a}||}$ the unit vector obtained by normalizing ${\bf a}$, and ${\bf b}$ another vector. Then Eq. \eqref{eq:dot_2} shows that
\begin{align*}
	{\bf \hat{a}}\cdot {\bf b}=\frac{{\bf a}}{||{\bf a}||} \cdot {\bf b}=||{\bf b}||\cos \theta
\end{align*} 
is the component of ${\bf b}$ in the direction of ${\bf a}$. Multiplying by ${\bf \hat{a}}$, we get a vector parallel to ${\bf a}$ whose magnitude is the component of ${\bf b}$ along ${\bf a}$.

\begin{defn}[Vector projection]
	The vector $\text{proj}_{\bf a}{\bf b}\equiv ({\bf \hat{a}}\cdot {\bf b}){\bf \hat{a}}=||{\bf b}||\cos \theta {\bf \hat{a}}$ is called the \ti{projection of ${\bf b}$ onto ${\bf a}$}.
\end{defn}
Geometrically, the projection of ${\bf b}=\overrightarrow{PQ}$ onto ${\bf a}=\overrightarrow{PS}$ is the vector $\overrightarrow{PR}$ determined by dropping a perpendicular from $Q$ to the line $PS$. \footnote{Note that by using the definitions of the dot product and the norm of ${\bf a}$, one may produce many equivalent expressions for $\text{proj}_{\bf a}{\bf b}$:
\begin{align*}
	\text{proj}_{\bf a}{\bf b}&=(||{\bf b}||\cos \theta){\bf \hat{a}} \\
	&=({\bf \hat{a}}\cdot {\bf b}){\bf \hat{a}} \\
	&=\left(\frac{{\bf a}\cdot {\bf b}}{||{\bf a}||}\right)\frac{{\bf a}}{||{\bf a}||} \\
	&=\left(\frac{{\bf b}\cdot {\bf a}}{{\bf a}\cdot {\bf a}}\right){\bf a}
\end{align*}
The first of these is perhaps the easiest to remember, as it makes most transparent the relation to elementary right triangle trigonometry.}

\begin{center}
	\includegraphics[scale=0.5]{figures_mvc/vector_projection_def}
\end{center}
Physically, if ${\bf b}$ represents a force, then $\text{proj}_{\bf a}{\bf b}$ is the effective force in the ${\bf a}$ direction.

\begin{center}
	\includegraphics[scale=0.5]{figures_mvc/effective_force_box}
\end{center}	

It is often desirable to express a vector ${\bf b}$ as a sum of two orthogonal vectors. For instance, in mechanics we frequently decompose forces in this way so that we may treat a two-dimensional problem as two one-dimensional problems. We can easily express a vector ${\bf b}$ as such a sum of two vectors, one parallel to some nonzero vector ${\bf a}$ and one orthogonal to ${\bf a}$, in terms of the projection of ${\bf b}$ along ${\bf a}$:
\begin{equation}
\begin{split}
	{\bf b}&={\bf b}_{\parallel}+{\bf b}_{\perp} \\
	&=\text{proj}_{\bf a}{\bf b}+({\bf b}-\text{proj}_{\bf a}{\bf b}).
\end{split}
\end{equation}

\begin{center}
	\includegraphics[scale=0.5]{figures_mvc/sum_of_orthogonal_vectors}
\end{center}

\begin{exercise}
	Express ${\bf b}=2{\bf e}_1+{\bf e}_2-3{\bf e}_3$ as the sum of a vector parallel to ${\bf a}=3{\bf e}_1-{\bf e}_2$ and a vector orthogonal to ${\bf a}$.
\end{exercise}

{\color{red} \flushleft {\bf Solution:}
Since ${\bf \hat{a}}\equiv \frac{{\bf a}}{||{\bf a}||}=\frac{3{\bf e}_1-{\bf e}_2}{\sqrt{10}}$, we can write ${\bf b}={\bf b}_{\parallel}+{\bf b}_{\perp}$ with  
\begin{align*}
	{\bf b}_{\parallel}&=({\bf \hat{a}}\cdot {\bf b}){\bf {\hat{a}}}=\frac{1}{2}(3{\bf e}_1-{\bf e}_2)=\frac{1}{2}{\bf a} \\
	{\bf b}_{\perp}&={\bf b}-{\bf b}_{\parallel}=\frac{1}{2}{\bf e}_1+\frac{3}{2}{\bf e}_2-3{\bf e}_3.
\end{align*}}

\subsubsection{The cross product}
There is another vector product which is important in physics, known as the \ti{cross product}. Unlike the dot product, which took two vectors and returned a \ti{scalar}, the cross product takes two vectors and returns another \ti{vector}.

The physical motivation for the cross product is that of \ti{torque}, which is a force that causes a rotation. Suppose we wish to tighten a bolt using a wrench. Applying a force ${\bf F}$ to the handle of the wrench produces a torque which acts along the axis of the bolt to drive the bolt forward. The magnitude of the torque depends on three things:
\begin{enumerate}
	\item The magnitude of ${\bf F}$. That is, how hard we push.
	\item How far out on the handle we apply the force. We denote the vector from the bolt to the point on the handle where the force is applied by ${\bf r}$. This vector is called the \ti{lever arm}. So, the magnitude of the torque also depends on the magnitude of the lever arm.
	\item Finally, the magnitude of the torque depends on the angle with respect to the axis of the handle at which the force is applied. \footnote{The angle we refer to here will be taken to be the \ti{smaller} of the two angles between the force vector and the lever arm.} The torque is zero when this angle is zero (since a force directed \ti{along} the axis of the wrench causes no rotation) and the torque is maximized then the force is perpendicular to the axis of the wrench.
\end{enumerate}
Denoting the torque by $\vec{\tau}$, one observes that 
\begin{equation}\label{eq:magnitude_of_torque}
	||\vec{\tau}||=||{\bf r}||\ ||{\bf F}||\ \sin \theta
\end{equation}
where, again, $\theta$ is the \ti{smaller} of the two angles between ${\bf r}$ and ${\bf F}$.
\begin{center}
	\includegraphics[scale=0.5]{figures_mvc/torque_wrench}
\end{center}
This gives the magnitude of $\vec{\tau}$. Notice that if we take the fingers of our \ti{right} hand and point then along the direction of ${\bf r}$ and then curl them toward ${\bf F}$ through the angle $\theta$ between ${\bf r}$ and ${\bf F}$ (arranged so that their initial points coincide), then our thumb points in the direction of $\vec{\tau}$. This is called the \ti{right-hand rule}. Letting ${\bf n}$ be a unit vector in the direction of $\vec{\tau}$ we have therefore found
\begin{equation}
	\vec{\tau}=||{\bf r}|| \ ||{\bf F}|| \ \sin \theta{\bf n}.
\end{equation}

\begin{example}
The magnitude of the torque exerted by the force ${\bf F}$ about the pivot point $P$ below 

\begin{center}
	\includegraphics[scale=0.5]{figures_mvc/torque_example}
\end{center}

is 
\begin{align*}
	||\vec{\tau}||=||\overrightarrow{PQ}|| \ ||{\bf F}|| \sin \theta = (3)(20)\sin 70^\circ \approx 56.4 \text{ ft-lb}.
\end{align*}
By the right hand rule, the torque is perpendicular to the plane  containing ${\bf r}$ and ${\bf F}$, and pointing \ti{out} of the page (rather than \ti{into} the page).
\end{example}


\begin{defn}[Cross product]
The \ti{cross product} of two nonzero vectors ${\bf a}$ and ${\bf b}$ is the vector
\begin{equation}
	{\bf a} \times {\bf b}=||{\bf a}|| \ ||{\bf b}|| \ \sin \theta{\bf n}
\end{equation}
where $\theta$ is the angle between ${\bf a}$ and ${\bf b}$ and ${\bf n}$ is a unit vector in the direction determined by the right-hand rule (that is, upon directing the fingers of ones right hand in the direction of ${\bf a}$ and curling them toward ${\bf b}$ through $\theta$, then ones thumb points in the direction of ${\bf n}$ and therefore $\overrightarrow{\tau}$). 

If either ${\bf a}$ or ${\bf b}$ is the zero vector, then we define ${\bf a} \times {\bf b}$ to be ${\bf 0}$.
\end{defn}

If ${\bf a}$ and ${\bf b}$ are both nonzero and if $\theta \neq 0$, then ${\bf a}$ and ${\bf b}$ define a plane, and ${\bf n}$ is perpendicular to this plane. The sense of ${\bf n}$ is the one determined by the right hand rule. 
\begin{center}
	\includegraphics[scale=0.5]{figures_mvc/cross_product}
\end{center}
Thus, 
\begin{center}
	\ti{The cross product ${\bf a} \times {\bf b}$ is a vector perpendicular to both ${\bf a}$ and ${\bf b}$}.
\end{center}

Note that if $\theta=0$ or $\pi$ (that is, if ${\bf a}$ and ${\bf b}$ are parallel, which means they don't determine a plane), then ${\bf a} \times {\bf b}=0$ . Conversely, if ${\bf a}$ and ${\bf b}$ are both non-zero, then we see from the definition that ${\bf a} \times {\bf b}=0$ only if ${\bf a}$ and ${\bf b}$ are parallel. Hence, we have proved the following proposition.

\begin{prop}[Parallel vectors]
	Two nonzero vectors ${\bf a}$ and ${\bf b}$ are parallel if and only if 
	\begin{align*}
		{\bf a} \times {\bf b}=0.
	\end{align*}
\end{prop}

Note that, geometrically, the magnitude of the cross product $||{\bf a} \times {\bf b}||=||{\bf a}|| \ ||{\bf b}||\sin\theta$ is the area of the parallelogram with adjacent sides ${\bf a}$ and ${\bf b}$:
\begin{center}
	\includegraphics[scale=0.5]{figures_mvc/cross_product_area_of_parallelogram}
\end{center}

Let us now look at the properties of the cross product. Despite its usefulness in physics, it turns out to preserve almost none of the properties of ordinary numerical multiplication. In particular, the cross product is neither commutative nor associative:


\begin{enumerate}[(1)]
	\item ${\bf a} \times {\bf b} \neq {\bf b}  \times {\bf a}$
	\item ${\bf a} \times ({\bf b} \times {\bf c}) \neq ({\bf a} \times {\bf b}) \times {\bf c}$
\end{enumerate}	

Regarding the first of these, if we reverse ${\bf a}$ and ${\bf b}$ in the definition, the only difference is that we rotate ${\bf b}$ into ${\bf a}$ rather than ${\bf a}$ into ${\bf b}$, so the right and rule gives
\begin{equation}
	{\bf a} \times {\bf b}=-{\bf b} \times {\bf a}.
\end{equation}
One the other hand, for the second of these, not only are the magnitudes of ${\bf a} \times ({\bf b} \times {\bf c})$ and $({\bf a} \times {\bf b}) \times {\bf c}$ not necessarily equal, these vectors do not even have to lie in the same plane! \footnote{Applying the definition of each cross product, one finds that ${\bf a} \times ({\bf b} \times {\bf c})$ is parallel to the plane determined by ${\bf b}$ and ${\bf c}$, while $({\bf a} \times {\bf b}) \times {\bf c}$ is parallel to the plane determined by ${\bf a}$ and ${\bf b}$. In particular, there is no reason why the ${\bf a}{\bf b}$-plane and the ${\bf b}{\bf c}$-plane should be the same, since three vectors don't have to lie in the same plane.}

However, vector and scalar distributive laws \ti{do} hold for the cross product:
\begin{prop}[Distributive properties of the cross product]
	\begin{enumerate}[(1)]\hspace{10cm}
	\item $(r{\bf a}) \times (s{\bf b})=(rs)({\bf a} \times {\bf b})$,
	\item ${\bf a} \times ({\bf b} + {\bf c})={\bf a} \times {\bf b}+{\bf a} \times {\bf c}$.
\end{enumerate}
\end{prop}

\begin{pf}
\begin{enumerate}[(1)]\hspace{10cm}
	\item We can verify this formula by applying the definition of the cross product to both sides of the equation:
	\begin{align*}
		||(r{\bf a}) \times (s{\bf b})||&=||(r{\bf a}) || \ ||(s{\bf b})||\sin \theta  \\&=|r| \ |s | \ ||{\bf a} || \ ||{\bf b} || \sin \theta \\
		&=|rs| \ ||{\bf a} || \ ||{\bf b} || \sin \theta \\
		&=||(rs)({\bf a}\times {\bf b})||
	\end{align*}
where we have used the fact that the angle between ${\bf a}$ and ${\bf b}$ is the same as the angle between $r{\bf a}$ and $s{\bf b}$.
	\item To derive this formula, we construct ${\bf a} \times {\bf b}$ in a new way. Draw ${\bf a}$ and ${\bf b}$ from the common point $O$ and construct a plane perpendicular to ${\bf a}$ at $O$.
\begin{center}
	\includegraphics[scale=0.5]{figures_mvc/distributivity_of_the_cross_product_1}
\end{center}
We then project ${\bf b}$ orthogonally onto $M$, yielding a vector ${\bf b}'$ with length $||{\bf b}||\sin \theta$. Finally, we rotate ${\bf b}'$ $90^\circ$ about ${\bf a}$ in the positive sense to produce a vector ${\bf b}''$, and then multiply ${\bf b}''$ by the length of ${\bf a}$. The resulting vector $||{\bf a}||{\bf b}''$ is equal to ${\bf a} \times {\bf b}$ since it has the same direction as ${\bf a} \times {\bf b}$ by construction and 
\begin{align*}
	||{\bf a}|| \ ||{\bf b}''||&=||{\bf a}|| \ ||{\bf b}'|| \ \text{ (since ${\bf b}''$ and ${\bf b}'$ are related by a rotation)} \\
	&=||{\bf a}|| \ ||{\bf b}||\sin\theta \ \text{ (since ${\bf b}'$ is the projection of ${\bf b}$ onto $M$)} \\
	&=||{\bf a} \times {\bf b}||. 
\end{align*}
Now each of these three operations, namely,
\begin{enumerate}
\item projection onto $M$,
\item rotation about ${\bf a}$ through $90^\circ$, 
\item multiplication by the scalar $||{\bf a}||$,
\end{enumerate}
when applied to a triangle whose plane is not parallel to ${\bf a}$, will produce another triangle. \footnote{If the triangle is in a plane parallel to ${\bf a}$ then projection onto $M$ will give a line segment, not a triangle.} If we start with a triangle whose sides are ${\bf b}, {\bf c},$ and ${\bf b}+{\bf c}$ and apply these three steps, we successively obtain
\begin{center}
	\includegraphics[scale=0.5]{figures_mvc/distributivity_of_the_cross_product_2}
\end{center}
\begin{enumerate}
\item a triangle whose sides are ${\bf b}', {\bf c}',$ and $({\bf b}+{\bf c})'$ satisfying the vector equation
	\begin{align*}
		{\bf b}'+{\bf c}'=({\bf b}+{\bf c})';
	\end{align*}
\item a triangle whose sides are ${\bf b}'', {\bf c}'',$ and $({\bf b}+{\bf c})''$ satisfying the vector equation
	\begin{align*}
		{\bf b}''+{\bf c}''=({\bf b}+{\bf c})'';
	\end{align*}
	and, finally,
\item a triangle whose sides are $||{\bf a}||{\bf b}'',$ $||{\bf a}||{\bf c}''$, and $||{\bf a}({\bf b}+{\bf c})''$ satisfying the vector equation
	\begin{equation}\label{eq:distributive_cross_product}
		||{\bf a}||{\bf b}''+||{\bf a}||{\bf c}''=||{\bf a}||({\bf b}+{\bf c})''.
	\end{equation}
\end{enumerate}
But we have shown above that $||{\bf a}||{\bf b}''={\bf a} \times {\bf b}$, $||{\bf a}||{\bf c}''={\bf a} \times {\bf c}$, and $||{\bf a}||({\bf b}+{\bf c})''={\bf a} \times ({\bf b}+{\bf c})$,
and therefore Eq. \eqref{eq:distributive_cross_product} is equivalent to 
\begin{align*}
	{\bf a} \times {\bf b}+{\bf a} \times {\bf c}={\bf a} \times ({\bf b} + {\bf c}),
\end{align*}
which is the property we wanted to establish.
\end{enumerate}	
\end{pf}

The cross product takes a useful form in  Cartesian coordinates. Let us first work out the various cross products of pairs of the unit vectors ${\bf e}_1, {\bf e}_2, {\bf e}_3$. Applying the definitions of the cross product and the ${\bf e}_i$s, we find 
\begin{align*}
	{\bf e}_i \times {\bf e}_j=\epsilon_{ijk}{\bf e}_k
\end{align*}
for $i,j,k \in \{1,2,3\}$, where
\begin{align*}
	\epsilon_{ijk}=\begin{cases}
		1, \text{ if $(ijk)$ is an even permutation of $(123)$} \\
		-1, \text{ if $(ijk)$ is an odd permutation of $(123)$} \\
		0, \text{ otherwise}
	\end{cases}
\end{align*}
Now let
\begin{align*}
	{\bf a}&=a_1{\bf e}_1+a_2{\bf e}_2+a_3{\bf e}_3 \\
	{\bf b}&=b_1{\bf e}_1+b_2{\bf e}_2+b_3{\bf e}_3 \\
\end{align*} 
We compute ${\bf a} \times {\bf b}$ using the distributive property above and then use the cross products of the ${\bf e}_i$s worked out above to simplify the 9 terms, ending up with \footnote{Those who have previously studied determinants can easily verify that this formula may be written as  
\begin{align*}
	{\bf a} \times {\bf b}=\begin{vmatrix}
		{\bf e}_1 & {\bf e}_2 & {\bf e}_3 \\
		a_1 & a_2 & a_3 \\
		b_1 & b_2 & b_3
	\end{vmatrix},
\end{align*} which is a useful mnemonic to remember the formula in Eq. \eqref{eq:a_cross_b}. We will study determinants in detail in a later unit.}
\begin{equation}\label{eq:a_cross_b}
	{\bf a} \times {\bf b}=(a_2b_3-a_3b_2){\bf e}_1-(a_1b_3-a_3b_1){\bf e}_2+(a_1b_2-a_2b_1){\bf e}_3.
\end{equation}
\fixme{Discuss even odd permutations of $(123)$ and how to remember the signs.}

\begin{exercise}
Compute ${\bf a} \times {\bf b}$ and ${\bf b} \times {\bf a}$ if ${\bf a}=(2,1,1)$ and ${\bf b}=(-4,3,1)$.	
\end{exercise}

{\color{red} \flushleft {\bf Solution:}
\begin{align*}
	{\bf a} \times {\bf b}&=\begin{vmatrix}
		{\bf e}_1 & {\bf e}_2 & {\bf e}_3 \\
		2 & 1 & 1 \\
		-4 & 3 & 1
	\end{vmatrix}
	=(-2,-6,10). \\
	{\bf b} \times {\bf a}&=-{\bf a} \times {\bf b}=(2,6,-10).
\end{align*}
}

\begin{exercise}
Find a vector perpendicular to the plane of $P(1,-1,0), Q(2,1,-1),$ and $R(-1,1,2)$.	
\end{exercise}

{\color{red} \flushleft {\bf Solution:}
Any three lines lie in a plane. We can construct two vectors in this plane:
\begin{align*}
	\overrightarrow{PQ}=(2-1,1-(-1),-1-0)=(1,2-1), \\
	\overrightarrow{PR}=(-1-1,1-(-1),2-0)=(-2,2,2). \\
\end{align*}
A vector perpendicular to this plane is then given by
\begin{align*}
	\overrightarrow{PQ} \times \overrightarrow{PR} =\begin{vmatrix}
		{\bf e}_1 & {\bf e}_2 & {\bf e}_3 \\
		1 & 2 & -1 \\ -2 & 2 & 2
	\end{vmatrix} = (6,0,6).
\end{align*}}

\begin{exercise}
	Find the area of the triangle with vertices $P(1,-1,0), Q(2,1,-1),$ and $R(-1,1,2)$.
\end{exercise}
{\color{red} \flushleft {\bf Solution:}
The area of the parallelogram determined by $P,Q,$ and $R$ is 
\begin{align*}
	||\overrightarrow{PQ} \times \overrightarrow{PR}||=||(6,0,6)||=6\sqrt{2}.
\end{align*}
The triangles area is half of this, or $3\sqrt{2}$.
\begin{center}
	\includegraphics[scale=0.5]{figures_mvc/triangle_area_cross_product}
\end{center}
}

\begin{exercise}
Find a unit vector perpendicular to the plane containing $P(1,-1,0), Q(2,1,-1),$ and $R(-1,1,2)$.	
\end{exercise}

{\color{red} \flushleft {\bf Solution:}
A unit vector perpendicular to the plane is given by
\begin{align*}
	{\bf n}=\frac{\overrightarrow{PQ} \times \overrightarrow{PR}}{||\overrightarrow{PQ} \times \overrightarrow{PR}||}=\frac{(6,0,6)}{6\sqrt{2}}=(1/\sqrt{2},0,1/\sqrt{2}).
\end{align*}}

\subsubsection{The triple scalar product}
Since ${\bf a} \times {\bf b}$ is a vector, note that the product $({\bf a} \times {\bf b}) \cdot {\bf c}$ is defined.

\begin{defn}[Triple scalar product]
	For any vectors ${\bf a}, {\bf b},$ and ${\bf c}$, the \ti{triple scalar product} is defined by
	\begin{align*}
		({\bf a} \times {\bf b}) \cdot {{\bf c}}=||{\bf a} \times {\bf b}|| \ ||{\bf c}|| \cos \theta
	\end{align*}
where $\theta$ is the angle between ${\bf a} \times {\bf b}$ and ${\bf c}$.
\end{defn}

Geometrically, the magnitude of ${\bf a} \times {\bf b} \cdot {{\bf c}}$ is the volume of the parallelepiped whose with adjacent sides ${\bf a}, {\bf b}$, and ${\bf c}$, the number $||{\bf a} \times {\bf b}||$ being the area of the base parallelogram and $||{\bf c}||\cos \theta$ the height of the parallelepiped.
\begin{center}
	\includegraphics[scale=0.5]{figures_mvc/triple_scalar_product_parallelepiped}
\end{center}
For this reason, the triple scalar product is also called the \ti{box product} of ${\bf a}, {\bf b},$ and ${\bf c}$.

Computing the triple scalar product in Cartesian coordinates, one can verify that 
\begin{equation}\label{eq:triple_scalar_product_formula}
\begin{split}
		({\bf a} \times {\bf b}) \cdot {{\bf c}}&=a_1(b_2c_3-c_3b_2)-a_2(b_1c_3-b_3c_1)+a_3(b_1c_2-c_1b_2)	 \\
		&=\begin{vmatrix}
		a_1 & a_2 & a_3 \\
		b_1 & b_2 & b_3 \\
		c_1 & c_2 & c_3  
	\end{vmatrix}. \\
\end{split}
\end{equation}

\begin{prop}[Cyclic symmetry of triple scalar product]\label{eq:cyclic_symmetry_of_triple_scalar_product}
For any three vectors ${\bf a}, {\bf b},$ and ${\bf c}$, the triple scalar product satisfies
\begin{align*}
	({\bf a} \times {\bf b}) \cdot {\bf c}=({\bf b} \times {\bf c}) \cdot {\bf a}=({\bf c} \times {\bf a}) \cdot {\bf b}.
\end{align*}	
\end{prop}

\begin{pf}
This is straightforward to verify from Eq. \eqref{eq:triple_scalar_product_formula}. For those who know about determinants, the determinant is unchanged under cyclic permutations of the rows.
\end{pf}

\begin{cor}[Interchange of dot and cross product in triple scalar product]
For any three vectors ${\bf a}, {\bf b},$ and ${\bf c}$, the triple scalar product satisfies
\begin{align*}
	({\bf a} \times {\bf b}) \cdot {\bf c}={\bf a}\cdot ({\bf b} \times {\bf c})
\end{align*}	
\end{cor}

\begin{pf}
This follows from Proposition \ref{eq:cyclic_symmetry_of_triple_scalar_product} along with commutativity of the dot product.	
\end{pf}

\begin{exercise}
Find the volume of the parallelepiped with adjacent sides ${\bf a}=(1,2,-1), {\bf b}=(-2,0,3)$, and ${\bf c}=(0,7,-4)$.	
\end{exercise}

{\color{red} \flushleft {\bf Solution:}
This is given by
\begin{align*}
	||{\bf a} \cdot ({\bf b} \times {\bf c})||=\vert \begin{vmatrix}
		1 & 2 & -1 \\ -2 & 0 & 3 \\ 0 & 7 & -4
	\end{vmatrix} \vert = \vert -23 \vert = 23.
\end{align*}}

\subsection{Equations of lines and planes}
\subsubsection{Lines in space}
The coordinate systems of analytic geometry allow us to consider geometric objects such as lines and planes in terms of vectors. These geometric ideas will give us valuable intuition later on in the course when we take a more abstract point of view toward vectors.

First let us recall that any two points define a line. Equivalently, we can also determine a line if we know one point on the line and the slope of the line. 

Let us now work in Cartesian coordinates. Suppose $L$ is a line passing through a point $P_0(x_0,y_0,z_0)$ and parallel to a vector ${\bf v}=v_1{\bf e}_1+v_2{\bf e}_2+v_3{\bf e}_3$. Now let $P(x,y,z)$ be any point in space. In which case will $P(x,y,z)$ to be on the line? This will be the case if the vector $\overrightarrow{P_0P}$ is parallel to ${\bf v}$, that is, if $\overrightarrow{P_0P}$ is a scalar multiple of ${\bf v}$. Therefore,
\begin{defn}[Vector equation for a line]
	The line through $P_0(x_0,y_0,z_0)$ and parallel of ${\bf v}$ is the set of all points $P(x,y,z)$ such that $\overrightarrow{P_0P}=t{\bf v}$, with $-\infty < t < \infty$. This equation is called the \ti{vector equation} of the line.
\end{defn}
In terms of Cartesian coordinates, the vector equation for the line becomes
\begin{align*}
	(x-x_0){\bf e}_1+(y-y_0){\bf e}_2+(z-z_0){\bf e}_3&=t(v_1{\bf e}_1+v_2{\bf e}_2+v_3{\bf e}_3) \\
	&=tv_1{\bf e}_1+tv_2{\bf e}_2+tv_3{\bf e}_3
\end{align*}
which implies
\begin{align*}
	(x-x_0-tv_1){\bf e}_1+(y-y_0-tv_2){\bf e}_2+(z-z_0-tv_3){\bf e}_3={\bf 0}
\end{align*}
and hence
\begin{equation}\label{eq:parametric_equations_line}
	x=x_0+tv_1, \hspace{0.25cm} y=y_0+tv_2, \hspace{0.25cm} z=z_0+tv_3.
\end{equation}
Thus, the vector equation of the line is equivalent to the three scalar equations in Eq. \eqref{eq:parametric_equations_line}, each of which is the usual equation for a line with slope $v_i$ in one variable $t$.

\begin{defn}[Parametric equations for a line]
	The standard parametrization of the line through $P_0(x_0,y_0,z_0)$ and parallel to ${\bf v}=v_1{\bf e}_1+v_2{\bf e}_2+v_3{\bf e}_3$ is given by
	\begin{align*}
		x=x_0+tv_1, \hspace{0.25cm} y=y_0+tv_2, \hspace{0.25cm} z=z_0+tv_3.
	\end{align*}
	These equations are called the (standard) \ti{parametric equations} for the line.
\end{defn}

\begin{exercise}
Find the parametric equations for the line through $(-2,0,4)$ and parallel to ${\bf v}=2{\bf e}_1+4{\bf e}_2-2{\bf e}_3$.	
\end{exercise}

{\color{red} \flushleft {\bf Solution:}
Plugging into Eq. \eqref{eq:parametric_equations_line} gives
\begin{align*}
	x=-2+2t, \hspace{0.25cm} y=4t, \hspace{0.25cm} z=4-2t.
\end{align*}}
\begin{center}
	\includegraphics[scale=0.5]{figures_mvc/parametrized_line_example_1}
\end{center}

\begin{example}
	Find parametric equations for the line through $P(-3,2,-3)$ and $Q(1,-1,4)$.
\end{example}

{\color{red} \flushleft {\bf Solution:}
The vector from $P$ to $Q$ is 
\begin{align*}
	\overrightarrow{PQ}&=(1-(-3), -1-2, 4-(-3)) \\
	&=(4,-3,7).
\end{align*}
We take this vector to be our ``${\bf v}$". The point $P_0$ could be either $P$ or $Q$. Arbitrarily choosing it to be $Q$, Eq. \eqref{eq:parametric_equations_line} gives
\begin{align*}
	x=1+4t, \hspace{0.25cm} y=-1-3t, \hspace{0.25cm} z=4+7t.
\end{align*}}

\begin{exercise}
Parametrize the line segment joining the points $P(-3,2,-3)$ and $Q(1,-1,4)$.	
\end{exercise}
{\color{red} \flushleft {\bf Solution:}
We have seen in the previous exercise that the  parametric equations 
\begin{align*}
	x=1+4t, \hspace{0.25cm} y=-1-3t, \hspace{0.25cm} z=4+7t.
\end{align*}
describe an infinite line containing $P$ and $Q$ when we take $-\infty < t < \infty$. To describe the line segment joining $P$ and $Q$, we simply restrict the domain of $t$. We see that the line passes through $P$ at $t=-1$ and $Q=0$. So the line segment joining $P$ and $Q$ is given by 

\begin{align*}
	x=1+4t, \hspace{0.25cm} y=-1-3t, \hspace{0.25cm} z=4+7t.
\end{align*}

with $-1 < t < 0$.}

\subsubsection{Planes in space}
Whereas a line is determined by any two points, a plane is determined by any \ti{three} non-collinear points. Similar to a line, which was equivalently determined by a single point and a slope, a plane can also be determined by a single point and a ``slope". The direction of the plane is determined by a normal vector ${\bf n}=n_1{\bf e}_1+n_2{\bf e}_2+n_3{\bf e}_3$. 

Let us now repeat the question we asked before: given a point $P_0(x_0,y_0,z_0)$ on the plane and a vector ${\bf n}$ normal to the plane, what is the condition for an arbitrary point in space $P(x,y,z)$ to lie on the plane? If $P$ lies in the plane, the $\overrightarrow{P_0P}$ is a vector lying in the plane. Then, since ${\bf n}$ is normal to the plane, we must have that ${\bf n} \cdot \overrightarrow{P_0P}=0$.

\begin{center}
	\includegraphics[scale=0.5]{figures_mvc/vector_equation_of_a_plane}
\end{center}

\begin{defn}[Vector equation for a plane]
	The \ti{vector equation} of the plane through $P_0(x_0,y_0,z_0)$ and normal to ${\bf n}=n_1{\bf e}_1+n_2{\bf e}_2+n_3{\bf e}_3$ is given by
	\begin{equation}
		{\bf n} \cdot \overrightarrow{P_0P}=0.
	\end{equation}
\end{defn}

As before, we can expand this equation in Cartesian coordinates to obtain

\begin{defn}[The component equation of a plane]
	The component equation of a plane through $P_0(x_0,y_0,z_0)$ and normal to ${\bf n}=n_1{\bf e}_1+n_2{\bf e}_2+n_3{\bf e}_3$ is given by
	\begin{equation}
		n_1(x-x_0)+n_2(y-y_0)+n_3(z-z_0)=0.
	\end{equation}
	This can be simplified to 
	\begin{equation}
		n_1x+n_2y+n_3z=c
	\end{equation}
	where $c=n_1x_0+n_2y_0+n_3z_0$.
\end{defn}

\begin{exercise}
Find an equation for the plane through $P_0(-3,0,7)$ perpendicular to ${\bf n}=(5,2,-1)$.	
\end{exercise}

{\color{red} \flushleft {\bf Solution:}
The component equation is 
\begin{align*}
	5(x-(-3))+2(y-0)+(-1)(z-7)=0,
\end{align*}
Simplifying, we obtain
\begin{align*}
	5x+15+2y-z+7&=0 \\
	5x+2y-z&=-22.
\end{align*}}

\begin{exercise}
Find the point where the line 
\begin{align*}
	x=\frac{8}{3}+2t, \hspace{0.25cm} y=2t, \hspace{0.25cm} z=1+t
\end{align*}	
intersects the plane $3x+2y+6z=6$.
\end{exercise}

{\color{red} \flushleft {\bf Solution:}
The point $(\frac{8}{3}+2t, 2t, 1+t)$ lies in the plane if its coordinates satisfy the equation of the plane; that is, if
\begin{align*}
	3(\frac{8}{3}+2t)+2(-2t)+6(1+t)=6
\end{align*}
This has a solution at $t=-1$, so the point of intersection is 
\begin{align*}
	(x,y,z)\vert_{t=-1}=(\frac{2}{3},2,0).
\end{align*}}

\begin{exercise}
\begin{enumerate}[(a)]
	\item Find a vector parallel to the line of intersection of the planes $3x-6y-2z=15$ and $2x+y-2z=5$.	
	\item Find parametric equations for the line in which these planes intersect.
\end{enumerate}
\end{exercise}

\subsection{$n$-dimensional space}
As we just have seen, the geometry of three-dimensional space is more complicated and harder to visualize than the geometry of one- or two-dimensional space. However, we have also seen that if we work in Cartesian coordinates, there is a remarkably simple structural resemblance between vectors in each of these spaces.

Consider a vector in each of these spaces:
\begin{enumerate}[(1)]
	\item If ${\bf v}=v_1{\bf e}_1$, then
		\begin{align*}
			||{\bf v}||=\sqrt{(v_1)^2}=|v_1|.
		\end{align*}
	\item If ${\bf v}=v_1{\bf e}_1+v_2{\bf e}_2$, then
		\begin{align*}
			||{\bf v}||=\sqrt{(v_1)^2+(v_2)^2.}
		\end{align*}
	\item If ${\bf v}=v_1{\bf e}_1+v_2{\bf e}_2+v_3{\bf e}_3$, then
		\begin{align*}
			||{\bf v}||=\sqrt{(v_1)^2+(v_2)^2+(v_3)^2}.
		\end{align*}
\end{enumerate} 
Similarly, in each dimension, we have seen that the basic operations of vector addition and scalar multiplication are given by

\begin{enumerate}[(1)]
	\item If ${\bf v}=v_1{\bf e}_1,{\bf w}=w_1{\bf e}_1,$ and $k$ a scalar, then
	\begin{align*}
		{\bf v}+{\bf w}&=(v_1+w_1){\bf e_1} \\
		k{\bf v}&=(kv_1){\bf e}_1.
	\end{align*}
	\item If ${\bf v}=v_1{\bf e}_1+v_2{\bf e}_2,{\bf w}=w_1{\bf e}_1+w_2{\bf e}_2,$ and $k$ a scalar, then
	\begin{align*}
		{\bf v}+{\bf w}&=(v_1+w_1){\bf e}_1+(v_2+w_2){\bf e}_2 \\
		k{\bf v}&=(kv_1){\bf e}_1+(kv_2){\bf e}_2.
	\end{align*}
	\item If ${\bf v}=v_1{\bf e}_1+v_2{\bf e}_2+v_3{\bf e}_3,{\bf w}=w_1{\bf e}_1+w_2{\bf e}_2+w_3{\bf e}_3,$ and $k$ a scalar, then
	\begin{align*}
		{\bf v}+{\bf w}&=(v_1+w_1){\bf e}_1+(v_2+w_2){\bf e}_2+(v_3+w_3){\bf e}_3 \\
		k{\bf v}&=(kv_1){\bf e}_1+(kv_2){\bf e}_2+(kv_3){\bf e}_3.
	\end{align*}
\end{enumerate}

We see that, once our rules and definitions are set, we don't have to worry about how difficult the geometry is. Except for the fact that we have an extra component to work with, there is no structural difference between working with vectors in two-dimensional space or three-dimensional space.

In fact, this construction has a natural generalization.

\begin{defn}[$n$-dimensional space]
	Consider the set of all $n$-tuples of real numbers $(x_1,x_2,\cdots,x_n)$. We denote this set by $\mathbb{R}^n$.
\end{defn}

\begin{example}
	The sets of all one-, two-, and three-tuples of real numbers, which we have been considering so far, are denoted $\mathbb{R}, \mathbb{R}^2,$ and $\mathbb{R}^3$, respectively.
\end{example}

Through the coordinate correspondence, $\mathbb{R}^n$ is geometrically obtained by introducing Cartesian coordinates for an $n$-dimensional Euclidean space. While the geometry of such a space is impossible to draw, we can define vectors in $\mathbb{R}^n$ by generalizing the definitions in $\mathbb{R}^3$:
\begin{itemize}
	\item A vector with its initial point at the origin $(0,0,\cdots,0)$ and terminal point at $(v_1,v_2,\cdots,v_n)$ will have components $(v_1,v_2,\cdots,v_n)$. Two vectors ${\bf v}$ and ${\bf w}$ are equal if 
	\begin{align*}
		v_1=w_1, \hspace{0.25cm} v_2=w_2, \hspace{0.25cm} \cdots \hspace{0.25cm} v_n=w_n
	\end{align*}
	The zero vector has components ${\bf 0}=(0,0,\cdots,0)$.
	\item We define vector addition and scalar multiplication as before: introducing unit vectors ${\bf e}_1, {\bf e}_2, \cdots, {\bf e}_2$ pointing along the coordinate axes, for two vectors ${\bf v}$ and ${\bf w}$ in $n$-dimensional space and $k$ any scalar,
\begin{align*}
	(v_1{\bf e}_1+\cdots+v_n{\bf e}_n)+(w_1{\bf e}_1+\cdots+w_n{\bf e}_n)&=(v_1+w_1){\bf e}_1+\cdots +(v_n+w_n){\bf e}_n \\
	k(v_1{\bf e}_1+\cdots+v_n{\bf e}_n)&=(kv_1){\bf e}_1+\cdots+(kv_n){\bf e}_n.
\end{align*}
\item The dot product of two vectors ${\bf v}$ and ${\bf w}$ in $\mathbb{R}^n$ is given by
\begin{align*}
	{\bf v}\cdot {\bf w}=v_1w_1+\cdots v_nw_n.
\end{align*}
\item The cross product is special to $\mathbb{R}^3$. There is no analog of the cross product for $\mathbb{R}^n$, when $n>3$. \footnote{It is possible to define a product which takes as input $n-1$ $n$-dimensional vectors and produces a vector perpendicular to each one of these, but we will not use such an operation in this course.}
\end{itemize}

We can use these definitions to show that the structural properties of vectors in $\mathbb{R}^3$ continue to hold for vectors in $\mathbb{R}^n$.

\begin{thm}[Properties of vectors in $\mathbb{R}^n$]
If ${\bf u}, {\bf v},$ and ${\bf w}$ are vectors in $\mathbb{R}^n$, and if $k$ and $m$ are scalars, then:
	\begin{enumerate}[(a)]
		\item ${\bf u}+{\bf v}={\bf v}+{\bf u}$
		\item $({\bf u}+{\bf v})+{\bf w}={\bf u}+({\bf v}+{\bf w})$
		\item ${\bf u}+{\bf 0}={\bf 0}+{\bf u}={\bf u}$
		\item ${\bf u}+(-{\bf u})={\bf 0}$
		\item $k({\bf u}+{\bf v})=k{\bf u}+k{\bf v}$
		\item $(k+m){\bf u}=k{\bf u}+m{\bf u}$
		\item $k(m{\bf u})=(km){\bf u}$
		\item $1{\bf u}={\bf u}$
		\item $0{\bf v}={\bf 0}$
		\item $k{\bf 0}={\bf 0}$
		\item $(-1){\bf v}=-{\bf v}$
	\end{enumerate}
\end{thm}

\begin{pf}
Exercise.	
\end{pf}

This theorem allows us to compute in $\mathbb{R}^n$ without explicitly writing out all the components, which can be extremely cumbersome. For instance, the previous theorem can be used to prove that if ${\bf x}+{\bf a}={\bf b}$, then ${\bf x}={\bf b}-{\bf a}$:
\begin{align*}
	{\bf x}+{\bf a}&={\bf b} \\
	({\bf x}+{\bf a})+(-{\bf a})&={\bf b}+(-{\bf a}) \\
	{\bf x}+({\bf a}+(-{\bf a}))&={\bf b}-{\bf a} \\
	{\bf x}+{\bf 0}&={\bf b}-{\bf a} \\
	{\bf x}&={\bf b}-{\bf a}.
\end{align*}

Finally, it is common for addition, subtraction, and scalar multiplication to be used in combination to form new vectors. For example, if ${\bf v}_1, {\bf v}_2,$ and ${\bf v}_3$ are vectors in $\mathbb{R}^n$, then in this way we can form the vectors
\begin{align*}
	{\bf u}=2{\bf v}_1+3{\bf v}_2+{\bf v}_3 \text{ and } {\bf w}=7{\bf v}_1-6{\bf v}_2+8{\bf v}_3.
\end{align*}
This motivates the following definition.

\begin{defn}[Linear combination]
	If ${\bf w}$ is a vector in $\mathbb{R}^n$, then ${\bf w}$ is said to be a \ti{linear combination} of the vectors ${\bf v}_1, {\bf v}_2, \cdots, {\bf v}_r$ in $\mathbb{R}^n$ if it can be expressed in the form
	\begin{align*}
		{\bf w}=k_1{\bf v}_1+k_2{\bf v}_2+\cdots +k_r{\bf v}_r
	\end{align*}
	where $k_1,k_2,\cdots,k_r$ are scalars. The scalars are called the \ti{coefficients} of the linear combination.
	
	In the case where $r=1$, this becomes ${\bf w}=k_1{\bf v}_1$, so a linear combination of a single vector is just a scalar multiple of that vector.
\end{defn}

\subsubsection{Applications of $n$-dimensional space}
\fixme{Finish.}

\subsubsection{3d Exercises}
\begin{defn}[Cartesian coordinates]
	The axis system in $\mathbb{E}^3$ is \ti{Cartesian} if the coordinate axes are mutually perpendicular and a common unit of distance (i.e., $|OQ_1|=|OQ_2|=|OQ_3|$) is used. The corresponding coordinates are called \ti{Cartesian coordinates}.
\end{defn}
A Cartesian coordinate system is shown below. \footnote{Cartesian coordinates are also often called \ti{rectangular coordinates} because the axes that define them meet at right angles.}
\begin{center}
	\includegraphics[scale=0.5]{figures_mvc/cartesian_coordinate_system}
\end{center}
The coordinate axes are labeled $x,y,$ and $z$, and are arranged such that if you take your right hand and curl your fingers from the positive $x$-axis toward the positive $y$-axis, your thumb points along the positive $z$-axis. Such a coordinate system is called \ti{right-handed}. \footnote{As you might have guessed, there are also left-handed coordinate systems. Choosing our coordinate system to be left- or right-handed is called choosing an \ti{orientation} for our coordinate system. We will discuss orientations more later. For now, just note that everything we do in the following could equivalently be formulated in terms of left-handed coordinate systems, so choosing a right-handed coordinate system for $\mathbb{E}^3$ is simply the standard convention.}

As discussed in the previous section, the coordinates of a point $P$ are the numbers at which the planes through $P$ perpendicular to the coordinate axes intersect these axes. These are labeled $\theta(P)=(x,y,z)$. 

Points on the $x$-axis have $y$- and $z$-coordinates equal to zero. That is, they have coordinates of the form $(x,0,0)$. Similarly points on the $y$- and $z$-axes have coordinates of the from $(0,y,0)$ and $(0,0,z)$, respectively.

The points in a plane perpendicular to the $x$-axis all have the same $x$-coordinate, $x_0$, this being the number at which the plane intersects that $x$-axis. The plane is then the set 
\begin{equation}
	P=\{(x,y,z) \in \mathbb{R}^3 : x=x_0\} 
\end{equation}
That is, the plane consists of all points having coordinates $(x_0,y,z)$, where $y$ and $z$ are any real numbers. The constraint $x=x_0$ is the equation of the plane perpendicular to the $x$-axis and intersecting it at $x_0$. Similar formulas hold for planes perpendicular to the $y$- and $z$-axes.

\begin{example}
The plane $x=2$ is the plane perpendicular to the $x$-axis at $x=2$. The plane $y=3$ is the plane perpendicular to the $y$-axis at $y=3$. The plane $z=5$ is the plane perpendicular to the $z$-axis at $z=5$. These planes are shown below, together with their intersection point $(2,3,5)$.
\end{example}

\begin{center}
	\includegraphics[scale=0.5]{figures_mvc/plane_235}
\end{center}

The planes $x=2$ and $y=3$ in the figure above intersect along a line parallel to the $z$-axis. This line is described by a pair of equations $L=\{(x,y,z) \in \mathbb{R}^3: x=2 \text{ and } y=3\}$.

\begin{exercise}
	Describe the other two lines of intersection in the figure above.
\end{exercise}

{\color{red} \flushleft {\bf Solution:}
The line running parallel to the $x$-axis is given by $$L_1=\{(x,y,z) \in \mathbb{R}^3: y=3 \text{ and } z=5\},$$ while the line running parallel to the $y$-axis is given by $$L_2=\{(x,y,z) \in \mathbb{R}^3: x=2 \text{ and } z=5\}.$$}

The three planes determined by the coordinate axes are the $xy$-plane, whose equation is $z=0$; the $yz$-plane, whose equation is $x=0$; and the $xz$-plane, whose equation is $y=0$. They meet at the origin, $(0,0,0)$.
	
The three coordinate planes $x=0, y=0,$ and $z=0$ divide space into eight cells called \ti{octants}. The octant in which the coordinates of all points are nonnegative is called the \ti{first octant}; there is no conventional numbering for the other seven octants.

\begin{exercise}
Write the coordinate equations and inequalities which define each of the following sets of points in space.
\begin{enumerate}[(a)]
	\item The half-space consisting of the points on and above the $xy$-plane.
	\item  This plane is parallel to the $yz$-plane and lying 3 units behind it.
	\item The second quadrant of the $xy$-plane.
	\item The first octant.
	\item The slab between the planes $y=-1$ and $y=1$, including these planes.
	\item The line in which the planes $y=-2$ and $z=2$ intersect.
\end{enumerate}	
\end{exercise}

{\color{red} \flushleft {\bf Solution:} 
\begin{enumerate}[(a)]
	\item $z \geq 0$,
	\item $x=-3$,
	\item $z=0, x \leq 0, z \geq 0$,
	\item $x \geq 0, y \geq 0, z \geq 0$,
	\item $-1 \leq y \leq 1$,
	\item $y=-2,z=2$.
\end{enumerate}}

\begin{exercise}
	What points $P(x,y,z)$ satisfy the equations
	\begin{align*}
		x^2+y^2=4 \text{ and }z=3?
	\end{align*}
\end{exercise}

{\color{red} \flushleft {\bf Solution:} These equations describe a circle of radius 2 lying in a plane parallel to the $xy$-plane and lying 3 units above it.}

\begin{center}
	\includegraphics[scale=0.5]{figures_mvc/circle_z=3}
\end{center}

\end{document}

